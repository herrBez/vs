\exercise{1.8}
{Provare che le equazioni della tabella 1.1 definiscono una funzione totale 
$\mathcal{A}:\aexp \rightarrow \states \rightarrow \Z$: 
	\begin{enumerate}
		\item Argomentare che è sufficiente provare che  
$\forall a \in \aexp$ e $\forall s \in \states$
		c'é esattamente un valore $\mathbf{v} \in \mathbf{Z}.\A{a} = 
\mathbf{v}$
		\item Usare l'induzione strutturale sulle espressioni 
aritmetiche per provare che è effettivamente così.
	\end{enumerate}
}
{
	\begin{enumerate}
		\item 
                \begin{itemize}
                  \item Per i casi base
                    \begin{itemize}
                            \item $\boxed{a = n \in Num}$
                            \item $\boxed{a = x \in Var}$
                    \end{itemize} 
                    abbiamo rispettivamente:
                    \begin{itemize}
                            \item $\boxed{\A{n} \mapsto \N{n}}$ \\
                              che $\forall s \in \states$ sappiamo essere un 
                              valore $\mathbf{v} \in \mathbf{Z}$ univoco, poichè 
                              per definizione $\mathcal{N}:\mathbf{Num} 
                              \rightarrow \mathbf{Z}$ è una funzione (totale) 
                              iniettiva.
                            \item $\boxed{\A{x} \mapsto s\myspace x}$ \\
                              poichè $s:\var \rightarrow \Z$ è una funzione 
                              (totale) allora $\forall s \in \states$ 
                              alla variabile $x$ verrà associato un unico valore 
                              $\mathbf{v} \in \Z$.
			
		  \item Per le espressioni composite \\
                        Ogni espressione composita è formata da un operatore 
                        binario e due sottoespressioni aritmetiche $a_1$ e 
                        $a_2$. $a_1$ ed $a_2$ vengono associate ad un unico 
                        valore $\mathbf{v_1}, \mathbf{v_2} \in \Z$.
			Per definizione gli operatori semantici $+, -, *$ 
                        prendono in input due valori e ne restituiscono 
                        rispettivamente la somma, la differenza e il prodotto. 
                        Gli operatori semantici restituiscono in tutti i
                        casi dei valori, l'asserto è dimostrato.
	        \end{itemize}	
			
		\end{itemize}
			\item Procedo per induzione strutturale su \aexp, 
                              formalizzando quanto argomentato in precedenza.
			\begin{table}[h!]
				\begin{center}
			\begin{tabular}{| l| L | p{0.4\linewidth} |}
				\hline
				\textbf{Casi Base} & \A{n}=\N{n}=\mathbf{v} & 
				è banalmente vero perchè per definizione \Ncal 
                                associa ad un numerale un (unico) numero intero.\\
				& &\\
				& \A{x} = s \myspace x = \mathbf{v}&$s$ è una 
                                funzione che ha come dominio tutte le variabili,
                                quindi ritorna un valore univoco $\mathbf{v}$ 
                                dipendentemente dallo stato.\\
				& & \\
				\hline
				& &\\
				\textbf{Casi Compositi} & 
                                \A{a_1 + a_2} = \A{a_1} + \A{a_2} & 
                                Definizione di \A{a_1 + a_2}\\
				& = \mathbf{v_1} + \mathbf{v_2}  &  
                                Ipotesi induttiva applicata a \A{a_1} e \A{a_2}\\
				& = \mathbf{v} & 
                                Definizione dell'operatore semantico\\
				& & \\
				& \A{a_1 - a_2} = \A{a_1} - \A{a_2} & 
                                Definizione di \A{a_1 - a_2}\\
				& = \mathbf{v_1} - \mathbf{v_2}  &  
                                Ipotesi induttiva applicata a \A{a_1} e \A{a_2}\\
				& = \mathbf{v} & 
                                Definizione dell'operatore semantico\\
				& & \\
				& \A{a_1 * a_2} = \A{a_1} * \A{a_2} & 
                                Definizione di \A{a_1 * a_2}\\
				& = \mathbf{v_1} * \mathbf{v_2}  & 
                                Ipotesi induttiva applicata a \A{a_1} e \A{a_2}\\
				& = \mathbf{v} & 
                                Definizione dell'operatore semantico\\
				& & \\
				\hline
			\end{tabular}
				
				\end{center}
			\end{table}
			
	\end{enumerate}
}
