\newcommand{\lt}{\ensuremath{\sqsubseteq}}

\exercise{Esercizio 15}
{
  Sia $D$ un CPO e $f: D \to D$ una funzione continua. Il teorema di
  Knaster-Tarski assicura che il minimo punto fisso di $f$ esiste.
  Fornire un controesempio al fine di mostrare che, nelle stesse condizioni, il
  massimo punto fisso di $f$ potrebbe non esistere.
}
{
}

Sia $\mathbb{Z}$ l'insieme dei numeri interi e $J$ l'insieme delle funzioni
parziali da $\mathbb{N}$ a $\mathbb{Z}$.

Sia \lt{} una relazione definita su $J$ nel seguente modo:
$$
a \lt b \iff |Dom(a)| \leq |Dom(b)|
$$

, dove $Dom(f)$ è il dominio di definizione di una funzione $f$.

Sia $g: \mathbb{N} \hookrightarrow{} \mathbb{Z}$ la seguente funzione:
$$
g_n\ x =
\begin{cases}
      x & se\ x,n\ \text{dispari} \land 0 \leq x \leq n \\
     -x & se\ x,n\ \text{pari}    \land 0 \leq x \leq n \\
     \undefDS & altrimenti
   \end{cases}
$$

Si vuole dunque dimostrare che $\exists{f} : J \to J$ per cui
non esiste massimo punto fisso. Una possibile $f$ può essere la seguente:
$$
f\ g_n =
\begin{cases}
      \bot    & se\ n < 0 \\
      g_{n+1} & se\ n \geq 0
   \end{cases}
$$

La discussione segue mostrando che $f$, pur essendo effettivamente monotona,
non ha massimo punto fisso.

% greatest upper bound: any funzione totale

\paragraph{Monotonia} $f$ è monotona:
\begin{itemize}
  \item se $g_n$ è t.c. $n < 0$, allora per riflessività $g_n \lt \bot$ poichè
    guardando la definizione di $g$ ci si accorge facilmente che quando $n < 0$
    non vi è alcun elemento in $\mathbb{N}$ per cui $g_n$ è definita;
  \item in caso contrario, $f$ restituisce $g$ con indice aumentato di $1$. Ma,
    come è facilmente visibile dalla definizione di $g$,
    $|Dom(g_{n+1})| \leq |Dom(g_n)| + 1$. Prendendo quindi $g_i \lt g_j$ in
    $J$, si ha
    $f(g_i) \lt f(g_j)$ poichè
    $$
    |Dom(g_i)| \leq |Dom(g_j)|
    \implies
    |Dom(g_i) + 1| \leq |Dom(g_j) + 1|
    \implies
    |Dom(g_{i+1})| \leq |Dom(g_{j+1})|
    $$
\end{itemize}

\paragraph{Minimo punto fisso}
La funzione $f$ ha un minimo punto fisso: infatti, prendendo l'elemento
``\textit{bottom}'' di $f$ (ovvero qualsiasi $g_n$ avente $n$ negativo) e
iterando l'applicazione di $f$ si ha
$$
f\ g_n = \bot
$$

Chiaramente, essendo $\bot$ la funzione indefinita su ogni input, $\bot$ è pure
l'elemento minimo di $J$\footnote{questo poichè $|Dom(\bot)| = 0$}.

Ma allora, essendo $\bot$ un punto fisso ed il minimo elemento di $J$, $\bot$ è
il \textbf{minimo} punto fisso di $J$.

\paragraph{Massimo punto fisso} Dalla definizione di $f$ è possibile intuire
che non esiste massimo punto per $g_n$ con $n \geq 0$: ripetendo infinite volte
l'iterazione, la funzione ottenuta continuerà ad ``oscillare'' tra l'identità
definita solo sui numeri dispari positivi e la funzione negazione definita solo
sui numeri pari negativi.

\undef{\lt}
