\newcommand{\lt}{\ensuremath{\sqsubseteq}}

\exercise{Esercizio 15}
{
  Sia $D$ un CPO e $f: D \to D$ una funzione continua. Il teorema di
  Knaster-Tarski assicura che il minimo punto fisso di $f$ esiste.
  Fornire un controesempio al fine di mostrare che, nelle stesse condizioni, il
  massimo punto fisso di $f$ potrebbe non esistere.
}
{
}

Sia $D \equiv \{\emptyset,\ \{0\},\ \{1\}\}$ con relazione di ordinamento
$\sqsubseteq$.

\setRel{D}{\sqsubseteq} è un CPO poichè:
\begin{itemize}
  \item \setRel{D}{\sqsubseteq} è un POSET:
  \begin{itemize}
    \item $\sqsubseteq$ è riflessivo su $D$;
    \item $\sqsubseteq$ è transitivo su $D$;
    \item $\sqsubseteq$ è antisimmetrico su $D$;
  \end{itemize}
  \item ogni catena in $D$ ha least upper bound:
  \begin{itemize}
    \item $\sqcup\{\emptyset\} = \{\emptyset\}$
    \item $\sqcup\{\{0\}\} = \{0\}$
    \item $\sqcup\{\{1\}\} = \{1\}$
    \item $\sqcup\{\emptyset, \{0\}\} = \{0\}$
    \item $\sqcup\{\emptyset, \{1\}\} = \{1\}$
  \end{itemize}
\end{itemize}

$\emptyset$ è l'elemento minimo di \setRel{D}{\sqsubseteq} poichè
è contenuto in sè stesso, in $\{0\}$ ed in $\{1\}$.
\\
Si consideri la seguente funzione (identità) $f: D \to D$:
$$
f\ x = x
$$

\paragraph{Continuità} $f$ è continua perchè: 
\begin{itemize}
  \item f è monotona, infatti preserva l'ordine parziale su $D$:
  \begin{itemize}
    \item riflessività:
    $$
      \forall x \in D.x \sqsubseteq x \Rightarrow f\ x \sqsubseteq f\ x
    $$
    \item transitività:
    $$
      (\forall x,y,z \in D.(x \sqsubseteq y\ \text{and}\ y \sqsubseteq z) \Rightarrow
      x \sqsubseteq z)
      \Rightarrow 
      ((f\ x \sqsubseteq f\ y\ \text{and}\ f\ y \sqsubseteq f\ z) \Rightarrow
      f\ x \sqsubseteq f\ z)
    $$
    \item antisimmetria:
     $$
      (\forall x,y \in D.(x \sqsubseteq y\ \text{and}\ y \sqsubseteq x) \Rightarrow
      x = y)
      \Rightarrow 
      ((f\ x \sqsubseteq f\ y\ \text{and}\ f\ y \sqsubseteq f\ x) \Rightarrow
      f\ x = f\ y)
    $$
  \end{itemize}
  \item Per ogni catena non vuota\footnote{vale anche per la catena vuota} $Y \in D$ vale:
    $$ 
      \sqcup\{f\ d\ |\ d \in Y\} = f(\sqcup Y)
    $$
    Ciò è banalmente vero perchè $f$ è l'identità.
\end{itemize}
\paragraph{Minimo punto fisso}
Essendo $f$ continua su CPO \setRel{D}{\sqsubseteq}, il teorema di Knaster-Tarski ci 
assicura l'esistenza del minimo punto fisso (è unico):  
$$
f\ \bot = f\ \emptyset = \emptyset
$$

\paragraph{Massimo punto fisso}
Ipotizziamo per assurdo che $f$ abbia massimo punto fisso $z \in D$.
Una condizione necessaria per l'esistenza di $z$ è che $D$ sia un reticolo 
(non necessariamente completo). Per verificare questa condizione basta 
dimostrare che esistono $greatest\ lower\ bound\ (\sqcap)$ e 
$least\ upper\ bound\ (\sqcup)$ per ogni coppia di elementi non comparabili.
Notiamo che $\not \exists\ \sqcup(\{1\},\{0\})$ perchè $D$ non è chiuso all'insù. 
Pertanto non esiste neppure $z$ e si ha quindi un assurdo.

\textbf{Prova alternativa.} È possibile vedere che non esiste massimo
punto fisso analizzando attentamente i punti fissi di $f$. Essendo $f$ la
funzione identità, i suoi punti fissi sono proprio gli elementi in $D$:
\begin{itemize}
  \item $\emptyset$
  \item $\{0\}$
  \item $\{1\}$
\end{itemize}
Ma non esiste un massimo punto fisso tra questi, poichè per ogni punto fisso
$f$ si ha che $\exists{d}\in{D}.\ f \not\sqsubseteq d$:
\begin{itemize}
  \item $\emptyset \not\sqsubseteq \{0\}$
  \item $\{0\} \not\sqsubseteq \{1\}$
  \item $\{1\} \not\sqsubseteq \{0\}$
\end{itemize}
\cvd
