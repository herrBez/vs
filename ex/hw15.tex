\newcommand{\lt}{\ensuremath{\sqsubseteq}}

\exercise{Esercizio 15}
{
  Sia $D$ un CPO e $f: D \to D$ una funzione continua. Il teorema di
  Knaster-Tarski assicura che il minimo punto fisso di $f$ esiste.
  Fornire un controesempio al fine di mostrare che, nelle stesse condizioni, il
  massimo punto fisso di $f$ potrebbe non esistere.
}
{
}

Sia $D \equiv \{\emptyset,\ \{0\},\ \{1\}\}$ con relazione di ordinamento
$\subseteq$.

\setRel{D}{\subseteq} è un CPO poichè:
\begin{itemize}
  \item $\subseteq$ è riflessivo su $D$;
  \item $\subseteq$ è transitivo su $D$;
  \item $\subseteq$ è antisimmetrico su $D$;
  \item ogni catena in $D$ ha least upper bound:
  \begin{itemize}
    \item $\sqcup\{\emptyset\} = \{\emptyset\}$
    \item $\sqcup\{\{0\}\} = \{0\}$
    \item $\sqcup\{\{1\}\} = \{1\}$
    \item $\sqcup\{\emptyset, \{0\}\} = \{0\}$
    \item $\sqcup\{\emptyset, \{1\}\} = \{1\}$
  \end{itemize}
\end{itemize}

Chiaramente, $\emptyset$ è l'elemento minimo di \setRel{D}{\subseteq} poichè
è contenuto in sè stesso, in $\{0\}$ ed in $\{1\}$.

Si consideri la seguente funzione (identità) $f: D \to D$:
$$
f\ x = x
$$

\paragraph{Monotonia} $f$ è monotona per riflessività di $\subseteq$.

\paragraph{Minimo punto fisso}
La funzione $f$ ha chiaramente minimo punto fisso, poichè
$$
f\ \bot = f\ \emptyset = \emptyset
$$

e in particolare $\emptyset$ è l'elemento minimo di $D$.

\paragraph{Massimo punto fisso}
La funzione $f$ non ha massimo punto fisso: infatti, sia $\{0\}$ che $\{1\}$
sono punti fissi, ma:
\begin{itemize}
  \item $\{0\} \not\subseteq \{1\}$;
  \item $\{1\} \not\subseteq \{0\}$; anche se
  \item $\emptyset \subseteq \{0\}$ e $\emptyset \subseteq \{1\}$.
\end{itemize}

\let\lt\undefined
