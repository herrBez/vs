\addtocontents{toc}{\protect\setcounter{tocdepth}{1}}
\exercise{2.21 (Composition Lemma)}
{
  Provare che \\
  \centerline{if \confSs{S_1}{s} $\Rar{k} s'$
  then \confSs{S_1; S_2}{s} $\Rar{k}$ \confSs{S_2}{s'}} \\
  Ovvero che l'esecuzione di $S_1$ non è influenzata dallo statement che la
  segue.
}
{
La seguente dimostrazione è sviluppata per induzione sulla lunghezza $k$ della
derivazione: \\

$\forall k,S_1,S_2,s,s'. if \mbox{ } \confSs{S_1}{s} \Rar{k} s' \mbox{ }
          then \mbox{ } \confSs{S_1; S_2}{s} \Rar{k} \confSs{S_2}{s'}$

\subsection{Caso base}

L'asserzione è vacuamente verificata per $k = 0$: non esistono derivazioni di
lunghezza nulla tali che $\confSs{S_1; S_2}{s} \Rar{0} s'$.

\subsection{Passo induttivo}

In questo caso assumiamo che la proprietà valga su ogni $k \leq k_0$ e si vuole
provare che questa vale su $k_0 + 1$.

Fissando $S_1, S_2, s, s'$ si ha dunque che è sempre valida l'ipotesi
$\confSs{S_1}{s} \Rar{k+1} s'$ o, con maggiore generalità, \\
$\confSs{S_1}{s} \Rar{1} \gamma{} \Rar{k} s'$.

Il ragionamento procede quindi considerando le varie strutture che $\gamma$ può
assumere:

\begin{itemize}
\item $\gamma$ è $s''$. Tuttavia tale configurazione non può essere
  ulteriormente derivata, perciò si conclude che in questo caso:
  \begin{itemize}
    \item $k = 0$; e
    \item $\confSs{S_1}{s} \Rar{1} \gamma{} = s'' = s'$.
  \end{itemize}

  Applicando la regola $[comp^2_{SOS}]$ si ha quindi verificata la correttezza
  della conseguenza $\confSs{S_1; S_2}{s} \Rar{k} \confSs{S_2}{s'}$.

\item $\gamma$ è \confSs{S_1'}{s''}, avendo
  quindi $\confSs{S_1}{s} \Rar{1} \confSs{S_1'}{s''} \Rar{k} s'$. Per ipotesi
  induttiva, se $\confSs{S_1'}{s''} \Rar{k} s'$ allora \\
  $\confSs{S_1'; S_2}{s''} \Rar{k} \confSs{S_2}{s'}$. Per la regola
  $[comp^1_{SOS}]$, si ha:

\begin{prooftree}
  \AxiomC{$\confSs{S_1}{s} \Rar{1} \confSs{S_1'}{s''}$}
  \UnaryInfC{$\confSs{S_1; S_2}{s} \Rar{1} \confSs{S_1'; S_2}{s''}$}
\end{prooftree}

  Sapendo che $\confSs{S_1'; S_2}{s''} \Rar{k} \confSs{S_2}{s'}$ è dunque
  verificata la tesi $\confSs{S_1; S_2}{s} \Rar{k+1} \confSs{S_2}{s'}$.
\end{itemize}
}
\addtocontents{toc}{\protect\setcounter{tocdepth}{2}}
