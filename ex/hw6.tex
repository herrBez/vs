\newcommand{\exitifSOS}{\SOSRule{exitif}{}{SOS}}
\exercise{Esercizio 6}
{Assume that the language \while{} includes a new iterative command:
$$
\wbSE{b}{S}{b'}
$$		
where $b, b' \in \bexp$ and $S \in \stm$, whose informal semantics goes as follows:
this behaves like a standard while-loop $\wbS{b}{S}$ where at the end of any iteration
of the body $S$ of the Boolean expression $b'$ is evaluated and if $b'$ is evaluated
to true then the loop is terminated (with no change to the current state).
\begin{enumerate}[label=$\arabic*$.]
	\item Define the small step operational semantics of this new iterative command.
	\item Prove the following semantic equivalence:
	$$
	\wbSE{b}{S}{\neg b} \eqSOS \wbS{b}{S}
	$$
\end{enumerate}
}
{
	\begin{enumerate}
\item In analogia con quanto fatto per $\wbS{b}{S}$ definisco $\wbSE{b}{S}{b'}$ come unfolding:
$$
\wbSE{b}{S}{b'} \Rar{ } \ifABC{b}{(S; (\ifABC{b'}{\skipForFriends}{(\wbSE{b}{S}{b'})}))}{\skipForFriends} 
$$	
e chiamo questo assioma $\exitifSOS$.

Argomento brevemente perché la regola definita è quella cercata:
se la guardia $b$ viene valutata $\semtrue$ allora viene eseguito il corpo
$S$. Una volta terminata la valutazione di $S$ viene valutata la guardia $b'$ se è vera si esce dal corpo senza
cambiare lo stato ($\skipForFriends$) 
altrimenti viene eseguito un nuovo ciclo.
\item Devo dimostrare che:
$$
\forall s, s' : (\confSs{\wbSE{b}{S}{\neg b}}{s} \Rar{*} s' \iff  \confSs{\wbS{b}{S}}{s} \Rar{*} s')
$$
\begin{proof}
Dimostro le due implicazioni separatamente.

$$
\boxed{\implies}
$$
Assumo che $\confSs{\wbSE{b}{S}{\neg b}}{s} \Rar{*} s'$.
Quindi applicando l'assioma $\exitifSOS$ si ottiene:
$$
\confSs{\ifABC{b}{(S; (\ifABC{\neg b}{\skipForFriends}{(\wbSE{b}{S}{\neg b})}))}{\skipForFriends}}{s}
$$
ora in base alla valutazione di $b$ nello stato $s$ distinguo due casi:
\begin{itemize}
	\item $\boxed{\B{b} = \semfalse}$:
	$$
	\begin{array}{lll}
	\confSs{\ifABC{b}{(S; (\ifABC{\neg b}{\skipForFriends}{(\wbSE{b}{S}{\neg b})}))}{\skipForFriends}}{s} & \Rar{} & \ifffSOS \\
	\confSs{\skipForFriends}{s} & \Rar{}& \skipSOS \\
	s
	\end{array}
	$$
	Da cui segue per il determinismo delle regole SOS (\textcolor{red}{N.B. TODO si dovrebbe mostrare che anche aggiungendo
	$\exitifSOS$ il determinismo si preserva}) che $s' = s$.
	Applicando la stessa assunzione a $\confSs{\wbS{b}{S}}{s}$ otteniamo:
	$$
	\begin{array}{ll}
	\confSs{\wbS{b}{S}}{s} \Rar{}& \whileSOS \\
	\confSs{\ifABC{b}{(S; \wbS{b}{S})}{\skipForFriends}}{s} \Rar{} & \ifffSOS \\ 
	\confSs{\skipForFriends}{s} \Rar{}& \skipSOS \\
	s
	\end{array}
	$$
	che è proprio ciò che si voleva dimostrare.
	\item $\boxed{\B{b} = \semtrue}$
	Procedendo con lo sviluppo della derivazione si ottiene:
	$$
	\begin{array}{lll}
	\confSs{\ifABC{b}{(S; (\ifABC{\neg b}{\skipForFriends}{(\wbSE{b}{S}{\neg b})}))}{\skipForFriends}}{s} & \Rar{} & \ifffSOS \\
	\confSs{S; \ifABC{\neg b}{\skipForFriends}{(\wbSE{b}{S}{\neg b})}}{s} & \Rar{*} & s'\\
	\end{array}
	$$
	per via del determinismo.
	Ora applicando il lemma di decomposizione si ottiene che $\exists s''$:
	\begin{enumerate}
	\item $\confSs{S}{s} \Rar{*} s''$
	\label{hw6:Ssgoestos''}
	\item $\confSs{\ifABC{\neg b}{\skipForFriends}{(\wbSE{b}{S}{\neg b})}}{s''} \Rar{*} s''$
	\label{hw6:ifnegb}
	\end{enumerate}
	visto che è un $\texttt{if}$ devo distinguere due casi:
	\begin{itemize}
		\item $\boxed{\B{\neg b}'' = \semtrue, \text{ ovvero } \B{b}'' = \semfalse}$
		In questo caso ottengo che:
		$$
		\begin{array}{ll}
		\confSs{\ifABC{\neg b}{\skipForFriends}{(\wbSE{b}{S}{\neg b})}}{s''} & \ifttSOS \\
		\confSs{\skipForFriends}{s''} & \skipForFriends \\
		s''
		\end{array}
		$$
		da cui per il determinismo delle regole SOS segue che $\boxed{s'' \equiv s'}$. Ma procedendo
		a valutare $\confSs{\wbS{b}{S}}{s}$ ottengo:
		$$
		\begin{array}{ll}
		\confSs{\wbS{b}{S}}{s} \Rar{} & \whileSOS \\
		\confSs{\ifABC{b}{S; \wbS{b}{S}}{\skipForFriends}}{s} & \ifttSOS \\
		\confSs{S; \wbS{b}{S}}{s} \Rar{*} & \text{Lemma di composizione applicato a \ref{hw6:Ssgoestos''}} \\
		\confSs{\wbS{b}{S}}{s''} & \whileSOS \\
		\confSs{\ifABC{b}{S; \wbS{b}{S}}{\skipForFriends}}{s''} & \ifffSOS \\
		\confSs{\skipForFriends}{s''} \Rar{} & \skipSOS \\
		s''
		\end{array}
		$$
		che era proprio ciò che bisognava dimostrare.
		\item $\boxed{\B{\neg b}'' = \semfalse, \text{ ovvero } \B{b}'' = \semtrue}$
		Procedo a valutare \ref{hw6:ifnegb}:
		$$
		\begin{array}{ll}
		\confSs{(\ifABC{\neg b}{\skipForFriends}{(\wbSE{b}{S}{\neg b})})}{s''} & \ifffSOS\\
		\confSs{\wbSE{b}{S}{\neg b}}{s''} \Rar{*} s'
		\end{array}
		$$
		che per ipotesi induttiva implica che
		$$
		\confSs{\wbS{b}{S}}{s} \Rar{*} s'
		$$
	\end{itemize}
	
\end{itemize}
$$
\boxed{\impliedby}
$$
Assumo che $\confSs{\wbS{b}{S}}{s} \Rar{*} s'$. Applicando un unfolding ($\whileSOS$) ottengo:
$$
\confSs{\ifABC{b}{(S; \wbS{b}{S})}{\skipForFriends}}{s}
$$
In cui posso discriminare due casi in base a $\B{b}$:
\begin{itemize}
\item $\boxed{\B{b} = \semfalse}$
$$
\begin{array}{ll}
\confSs{\ifABC{b}{(S; \wbS{b}{S})}{\skipForFriends}}{s} & \ifffSOS\\
\confSs{\skipForFriends}{s} & \skipSOS\\
s & \\
\end{array}
$$
Da cui segue che $\boxed{s' \equiv{ } s}$
$$
\begin{array}{ll}
\confSs{\wbSE{b}{S}{\neg b}}{s} \Rar{ } & \exitifSOS\\
\confSs{\ifABC{b}{(S; (\ifABC{\neg b}{\skipForFriends}{(\wbSE{b}{S}{\neg b})}))}{\skipForFriends}}{s} \Rar{} & \ifffSOS \\
\confSs{\skipForFriends}{s} \Rar{}& \skipSOS\\
s & \\
\end{array}
$$
\item $\boxed{\B{b} = \semtrue}$
$$
\begin{array}{ll}
\confSs{\wbS{b}{S}}{s} \Rar{*} & \\ 
\confSs{\ifABC{b}{(S; \wbS{b}{S})}{\skipForFriends}}{s} \Rar{} & \ifttSOS\\
\confSs{S; \wbS{b}{S}}{s} & \\
\end{array}
$$	
Siccome le regole SOS sono deterministiche ottengo che
$$
\confSs{S; \wbS{b}{S}}{s} \Rar{*} s'
$$
e quindi per il lemma di decomposizione $\exists s''$:
\begin{enumerate}
\item 
$\confSs{S}{s} \Rar{*} s''$
\label{hw6:FattoA}
\item $\confSs{\wbS{b}{S}}{s''} \Rar{*} s' $
\label{hw6:FattoB}
\end{enumerate}
Inizio anche a sviluppare 
$$
\begin{array}{ll}
	\confSs{\wbSE{b}{S}{\neg b}}{s} \Rar{ } & \exitifSOS\\
	\confSs{\ifABC{b}{(S; (\ifABC{\neg b}{\skipForFriends}{(\wbSE{b}{S}{\neg b})}))}{\skipForFriends}}{s} \Rar{} & \ifttSOS \\
	\confSs{S; (\ifABC{\neg b}{\skipForFriends}{(\wbSE{b}{S}{\neg b})})}{s} & \\
\end{array}
$$
Usando il fatto \ref{hw6:FattoA}  e il lemma di composizione ottengo che:
$$
\begin{array}{ll}
\confSs{S; (\ifABC{\neg b}{\skipForFriends}{(\wbSE{b}{S}{b'})})}{s} \Rar{*} & \\
\confSs{\ifABC{\neg b}{\skipForFriends}{(\wbSE{b}{S}{b'})}}{s''} \Rar{}
\end{array}
$$
Ora invece di procedere come di consueto distinguo ulteriori due casi:
\begin{itemize}
\item $\boxed{\B{b}'' = \semfalse}$
Applicando questa ipotesi al fatto $\ref{hw6:FattoB}$ ottengo che $\boxed{s'' \equiv s'}$.
al contempo applicando questa ipotesi ottengo anche:
$$
\begin{array}{ll}
	\confSs{S; (\ifABC{\neg b}{\skipForFriends}{(\wbSE{b}{S}{b'})})}{s} \Rar{*} & \\
	\confSs{\ifABC{\neg b}{\skipForFriends}{(\wbSE{b}{S}{b'})}}{s''} \Rar{} & \ifttSOS \\
	\confSs{\skipForFriends}{s''} \Rar{ } s'' & \\
\end{array}
$$
quindi ho dimostrato che $\wbSE{b}{S}{\neg b} \Rar{*} s'$ perchè $s' \equiv s''$.
\item $\boxed{\B{b}'' = \semtrue}$	
In questo caso ottengo che
$$
\confSs{\wbS{b}{S}}{s''} \Rar{*} s'
$$
e a destra ottengo
$$
\confSs{\wbSE{b}{S}{\neg b}}{s''}
$$
ma quindi grazie all'ipotesi induttiva\footnote{Si guardi la parentesi matematica per una spiegazione.} ottengo che
$$
\confSs{\wbSE{b}{S}{\neg b}}{s''} \Rar{*} s'
$$
\paragraph{Parentesi matematica (Perchè l'induzione funziona)}

Il tutto può essere visto come induzione sul numero di passi di derivazione
della configurazione: $$\confSs{\wbS{b}{S}}{s} \Rar{n} s'$$
Sia $n$ il numero di passi necessari per arrivare ad $s'$, allora:
\begin{itemize}
\item Se $\boxed{\B{b} = \semfalse}$ $n = 3$.
\item Se $\boxed{\B{b} = \semtrue \land \B{b}'' = \semfalse}$, $n = 2 + l + 2 + 1$ dove $l$ è il numero
di passi per $\confSs{S}{s} \Rar{l} s$
\end{itemize}
che sono i casi base della dimostrazione.

Il passo induttivo, ovvero il caso in cui $\B{b} = \B{b}'' = \semtrue$
richiede almeno (Nel caso induttivo più piccolo):
$$
n = 2 + l + 2 + k + 2 + 1
$$
passi, 
con $k$ tale che $\confSs{\wbS{b}{S}}{s''} \Rar{k} s'''$.
Quindi in generale se $\confSs{\wbS{b}{S}}{s} \Rar{*} s'$ richiede $w$ iterazioni per essere calcolata:
$$
n = 2 + \left(\sum_{i=1}^{w} (l_i + 2)\right) + 1
$$
dove 
$$
\confSs{S}{s_{i-1}} \Rar{l_i} \confSs{S}{s_i}
$$
con $\boxed{s_0 = s}$ e $\boxed{s_w = s'}$. Naturalmente se $w = 0$
allora $s_0 = s_w = s = s'$.
Siccome $S$ deve contenere almeno un'istruzione, questa ha bisogno almeno di $1$ ($\forall i \leq w\ \text{ vale } l_i \geq 1$) passo per
essere derivato completamente (nel caso di $\skipForFriends$ e $x := a$), quindi 
ad ogni iterazione completata vengono rimossi almeno $(l_i + 2) \geq 3$ passi da eseguire, 
per cui dopo aver 'eseguito' un'iterazione può essere applicata l'ipotesi induttiva.
Nel caso specifico della dimostrazione
$$
\confSs{\wbS{b}{S}}{s} \Rar{2} \confSs{S; \wbS{b}{S}}{s} \Rar{l+2} \confSs{\wbS{b}{S}}{s''}
$$
cui posso applicare l'ipotesi induttiva.
\end{itemize}
\end{itemize}
\end{proof}
\end{enumerate}
}