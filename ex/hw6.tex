\newcommand{\exitifSOS}{\SOSRule{exitif}{}{SOS}}
\exercise{Esercizio 6}
{Si assuma che il linguaggio \while{} includa un nuovo comando iterativo:
$$
\wbSE{b}{S}{b'}
$$
dove $b, b' \in \bexp$ e $S \in \stm$, la cui semanticac informale è definita
come segue:
``si comporta come un while-loop standard $\wbS{b}{S}$ in cui, alla fine di
ogni iterazione del corpo $S$, l'espressione booleana $b'$ viene valutata e se
$b'$ è valutata a $\semtrue$ allora il loop termina (senza modifiche allo
stato corrente''.
\begin{enumerate}[label=$\arabic*$.]
	\item Si definisca la SOS di questo nuovo comando iterativo.
	\item Si provi la seguente equivalenza semantica:
	$$
	\wbSE{b}{S}{\neg b} \eqSOS \wbS{b}{S}
	$$
\end{enumerate}
}
{}
	\begin{enumerate}
\item In analogia con quanto fatto per $\wbS{b}{S}$ definisco $\wbSE{b}{S}{b'}$ come unfolding:
$$
\wbSE{b}{S}{b'} \Rar{ } \ifABC{b}{(S; (\ifABC{b'}{\skipForFriends}{(\wbSE{b}{S}{b'})}))}{\skipForFriends}
$$
e chiamo questo assioma $\exitifSOS$.

Argomento brevemente perché la regola definita è quella cercata:
se la guardia $b$ viene valutata $\semtrue$ allora viene eseguito il corpo
$S$. Una volta terminata la valutazione di $S$ viene valutata la guardia $b'$:
se è vera si esce dal corpo senza cambiare lo stato ($\skipForFriends$),
altrimenti viene eseguito un nuovo ciclo.

\item Devo dimostrare che:
$$
\forall s, s'. (\confSs{\wbSE{b}{S}{\neg b}}{s} \Rar{m} s'
  \iff
\confSs{\wbS{b}{S}}{s} \Rar{n} s')
$$
\begin{proof} \mbox{ }

\paragraph{Nota} La semantica SOS aumentata con la regola $\exitifSOS$ è ancora
deterministica per induzione strutturale: $\wbSE{b}{S}{b'}$ viene infatti
derivato\footnote{N.B. nessuna altra derivazione è possibile} come
$$
\ifABC{b}{(S; (\ifABC{b'}{\skipForFriends}{(\wbSE{b}{S}{b'})}))}{\skipForFriends}
$$
Ma questo termine evolverà in modo deterministico:
\begin{itemize}
  \item è noto che \texttt{if-then-else} e \texttt{skip} vengono derivati in
    modo deterministico;
  \item la sottoespressione $\wbSE{b}{S}{b'}$ viene derivata in modo
    deterministico per ipotesi induttiva.
\end{itemize}

\noindent(Fine nota) \vspace{1em} \\

L'esercizio viene svolto per induzione sulla lunghezza della derivazione
$m$ o $n$ a seconda del verso della doppia implicazione che si prende in
considerazione.

I casi $m = 0$, $n = 0$ vengono dati banalmente per veri poichè sono
vacuamente verificati.

$$
\boxed{\implies}
$$
Assumo che $\confSs{\wbSE{b}{S}{\neg b}}{s} \Rar{m} s'$.
Quindi applicando l'assioma $\exitifSOS$ si ottiene:
$$
\confSs{\ifABC{b}{(S; (\ifABC{\neg b}{\skipForFriends}{(\wbSE{b}{S}{\neg b})}))}{\skipForFriends}}{s}
$$
ora in base alla valutazione di $b$ nello stato $s$ distinguo due casi:
\begin{itemize}
	\item $\boxed{\B{b} = \semfalse}$:
	$$
	\begin{array}{lll}
	\confSs{\ifABC{b}{(S; (\ifABC{\neg b}{\skipForFriends}{(\wbSE{b}{S}{\neg b})}))}{\skipForFriends}}{s} & \Rar{} & \ifffSOS \\
	\confSs{\skipForFriends}{s} & \Rar{}& \skipSOS \\
	s
	\end{array}
	$$
	Da cui segue per il determinismo delle regole SOS che $s' \equiv s$.
	Applicando la stessa assunzione a $\confSs{\wbS{b}{S}}{s}$ otteniamo:
	$$
	\begin{array}{ll}
	\confSs{\wbS{b}{S}}{s} \Rar{}& \whileSOS \\
	\confSs{\ifABC{b}{(S; \wbS{b}{S})}{\skipForFriends}}{s} \Rar{} & \ifffSOS \\ 
	\confSs{\skipForFriends}{s} \Rar{}& \skipSOS \\
	s
	\end{array}
	$$
	che è proprio ciò che si voleva dimostrare.
	\item $\boxed{\B{b} = \semtrue}$
	Procedendo con lo sviluppo della derivazione si ottiene:
	$$
	\begin{array}{lll}
	\confSs{\ifABC{b}{(S; (\ifABC{\neg b}{\skipForFriends}{(\wbSE{b}{S}{\neg b})}))}{\skipForFriends}}{s} & \Rar{} & \ifttSOS \\
	\confSs{S; \ifABC{\neg b}{\skipForFriends}{(\wbSE{b}{S}{\neg b})}}{s} & \Rar{*} & s'\\
	\end{array}
	$$
	per via del determinismo e dell'ipotesi $\implies$.
	Ora applicando il lemma di decomposizione si ottiene che
  $\exists s''\in\states$:
	\begin{enumerate}
	\item $\confSs{S}{s} \Rar{*} s''$
	\label{hw6:Ssgoestos''}
	\item $\confSs{\ifABC{\neg b}{\skipForFriends}{(\wbSE{b}{S}{\neg b})}}{s''} \Rar{*} s'$
	\label{hw6:ifnegb}
	\end{enumerate}
	visto che è un $\texttt{if}$ devo distinguere due casi:
	\begin{itemize}
		\item $\boxed{\B{\neg b}'' = \semtrue, \text{ ovvero } \B{b}'' = \semfalse}$
		In questo caso ottengo che:
		$$
		\begin{array}{ll}
		\confSs{\ifABC{\neg b}{\skipForFriends}{(\wbSE{b}{S}{\neg b})}}{s''} & \ifttSOS \\
		\confSs{\skipForFriends}{s''} & \skipForFriends \\
		s''
		\end{array}
		$$
		da cui per il determinismo delle regole SOS segue che $\boxed{s'' \equiv s'}$. Ma procedendo
		a valutare $\confSs{\wbS{b}{S}}{s}$ ottengo:
		$$
		\begin{array}{ll}
		\confSs{\wbS{b}{S}}{s} \Rar{} & \whileSOS \\
		\confSs{\ifABC{b}{S; \wbS{b}{S}}{\skipForFriends}}{s} & \ifttSOS \\
		\confSs{S; \wbS{b}{S}}{s} \Rar{*} & \text{Lemma di composizione applicato a \ref{hw6:Ssgoestos''}} \\
		\confSs{\wbS{b}{S}}{s''} & \whileSOS \\
		\confSs{\ifABC{b}{S; \wbS{b}{S}}{\skipForFriends}}{s''} & \ifffSOS \\
		\confSs{\skipForFriends}{s''} \Rar{} & \skipSOS \\
		s''
		\end{array}
		$$
		che era proprio ciò che bisognava dimostrare.
		\item $\boxed{\B{\neg b}'' = \semfalse, \text{ ovvero } \B{b}'' = \semtrue}$
		Procedo a valutare \ref{hw6:ifnegb}:
		$$
		\begin{array}{ll}
		\confSs{(\ifABC{\neg b}{\skipForFriends}{(\wbSE{b}{S}{\neg b})})}{s''} & \ifffSOS\\
		\confSs{\wbSE{b}{S}{\neg b}}{s''} \Rar{m_1} s'
		\end{array}
		$$
    Sotto le stesse ipotesi, il termine \confSs{\wbS{b}{S}}{s} evolve come
    segue:
    $$
    \begin{array}{ll}
    \confSs{\wbS{b}{S}}{s} \Rar{} & \whileSOS \\
    \confSs{\ifABC{b}{S; \wbS{b}{S}}{\skipForFriends}}{s} & \ifttSOS \\
    \confSs{S; \wbS{b}{S}}{s} \Rar{*} & \text{Lemma di composizione applicato a \ref{hw6:Ssgoestos''}} \\
    \confSs{\wbS{b}{S}}{s''}
    \end{array}
    $$
		Siccome è necessario dimostrare la stessa tesi iniziale con premessa avente
    lunghezza di derivazione $m_1$ strettamente inferiore di $m$, possiamo
    applicare l'ipotesi induttiva (dove $n_1 \leq n$):
		$$
		\confSs{\wbS{b}{S}}{s''} \Rar{n_1} s'
		$$
    E quindi si ha:
    $$
    \confSs{\wbS{b}{S}}{s} \Rar{n} s'
    $$
	\end{itemize}
	
\end{itemize}
$$
\boxed{\impliedby}
$$
Assumo che $\confSs{\wbS{b}{S}}{s} \Rar{*} s'$. Applicando un unfolding ($\whileSOS$) ottengo:
$$
\confSs{\ifABC{b}{(S; \wbS{b}{S})}{\skipForFriends}}{s}
$$
In cui posso discriminare due casi in base a $\B{b}$:
\begin{itemize}
\item $\boxed{\B{b} = \semfalse}$
$$
\begin{array}{ll}
\confSs{\ifABC{b}{(S; \wbS{b}{S})}{\skipForFriends}}{s} & \ifffSOS\\
\confSs{\skipForFriends}{s} & \skipSOS\\
s & \\
\end{array}
$$
Da cui segue che $\boxed{s' \equiv{ } s}$
$$
\begin{array}{ll}
\confSs{\wbSE{b}{S}{\neg b}}{s} \Rar{ } & \exitifSOS\\
\confSs{\ifABC{b}{(S; (\ifABC{\neg b}{\skipForFriends}{(\wbSE{b}{S}{\neg b})}))}{\skipForFriends}}{s} \Rar{} & \ifffSOS \\
\confSs{\skipForFriends}{s} \Rar{}& \skipSOS\\
s & \\
\end{array}
$$
\item $\boxed{\B{b} = \semtrue}$
$$
\begin{array}{ll}
\confSs{\wbS{b}{S}}{s} \Rar{*} & \\ 
\confSs{\ifABC{b}{(S; \wbS{b}{S})}{\skipForFriends}}{s} \Rar{} & \ifttSOS\\
\confSs{S; \wbS{b}{S}}{s} & \\
\end{array}
$$	
Siccome le regole SOS sono deterministiche ottengo che
$$
\confSs{S; \wbS{b}{S}}{s} \Rar{*} s'
$$
e quindi per il lemma di decomposizione $\exists s''$:
\begin{enumerate}
\item 
$\confSs{S}{s} \Rar{*} s''$
\label{hw6:FattoA}
\item $\confSs{\wbS{b}{S}}{s''} \Rar{*} s' $
\label{hw6:FattoB}
\end{enumerate}
Inizio anche a sviluppare 
$$
\begin{array}{ll}
	\confSs{\wbSE{b}{S}{\neg b}}{s} \Rar{ } & \exitifSOS\\
	\confSs{\ifABC{b}{(S; (\ifABC{\neg b}{\skipForFriends}{(\wbSE{b}{S}{\neg b})}))}{\skipForFriends}}{s} \Rar{} & \ifttSOS \\
	\confSs{S; (\ifABC{\neg b}{\skipForFriends}{(\wbSE{b}{S}{\neg b})})}{s} & \\
\end{array}
$$
Usando il fatto \ref{hw6:FattoA}  e il lemma di composizione ottengo che:
$$
\begin{array}{ll}
\confSs{S; (\ifABC{\neg b}{\skipForFriends}{(\wbSE{b}{S}{b'})})}{s} \Rar{*} & \\
\confSs{\ifABC{\neg b}{\skipForFriends}{(\wbSE{b}{S}{b'})}}{s''}
\end{array}
$$
Ora invece di procedere come di consueto distinguo ulteriori due casi:
\begin{itemize}
\item $\boxed{\B{b}'' = \semfalse}$
Applicando questa ipotesi al fatto $\ref{hw6:FattoB}$ ottengo che $\boxed{s'' \equiv s'}$.
Al contempo applicando questa ipotesi ottengo anche:
$$
\begin{array}{ll}
	\confSs{S; (\ifABC{\neg b}{\skipForFriends}{(\wbSE{b}{S}{b'})})}{s} \Rar{*} & \\
	\confSs{\ifABC{\neg b}{\skipForFriends}{(\wbSE{b}{S}{b'})}}{s''} \Rar{} & \ifttSOS \\
	\confSs{\skipForFriends}{s''} \Rar{ } s'' & \\
\end{array}
$$
quindi ho dimostrato che $\wbSE{b}{S}{\neg b} \Rar{*} s'$ perchè $s' \equiv s''$.
\item $\boxed{\B{b}'' = \semtrue}$	
In questo caso ottengo che
$$
\confSs{\wbS{b}{S}}{s''} \Rar{*} s'
$$
e a destra ottengo
$$
\confSs{\wbSE{b}{S}{\neg b}}{s''}
$$
ma quindi, per induzione\footnote{Si guardi la parentesi matematica per una
spiegazione più dettagliata.} sulla lunghezza della derivazione ottengo che
$$
\confSs{\wbSE{b}{S}{\neg b}}{s''} \Rar{*} s'
$$
\end{itemize}
\end{itemize}
\end{proof}
\end{enumerate}
