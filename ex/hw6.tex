\newcommand{\exitifSOS}{\SOSRule{exitif}{}{SOS}}
\exercise{Esercizio 6}
{Assume that the language \while{} includes a new iterative command:
$$
\wbSE{b}{S}{b'}
$$		
where $b, b' \in \bexp$ and $S \in \stm$, whose informal semantics goes as follows:
this behaves like a standard while-loop $\wbS{b}{S}$ where at the end of any iteration
of the body $S$ of the Boolean expression $b'$ is evaluated and if $b'$ is evaluated
to true then the loop is terminated (with no change to the current state).
\begin{enumerate}[label=$\arabic*$.]
	\item Define the small step operational semantics of this new iterative command.
	\item Prove the following semantic equivalence:
	$$
	\wbSE{b}{S}{\neg b} \eqSOS \wbS{b}{S}
	$$
\end{enumerate}
}
{
	\begin{enumerate}
\item In analogia con quanto fatto per $\wbS{b}{S}$ definisco $\wbSE{b}{S}{b'}$ come unfolding:
$$
\wbSE{b}{S}{b'} \Rar{ } \ifABC{b}{(S; (\ifABC{b'}{\skipForFriends}{(\wbSE{b}{S}{b'})}))}{\skipForFriends} 
$$	
e chiamo questo assioma $\exitifSOS$.

Argomento brevemente perché la regola definita è quella cercata:
se la guardia $b$ viene valutata $\semtrue$ allora viene eseguito il corpo
$S$. Una volta terminata la valutazione di $S$ viene valutata la guardia $b'$ se è vera si esce dal corpo senza
cambiare lo stato ($\skipForFriends$) 
altrimenti viene eseguito un nuovo ciclo.
\item Devo dimostrare che:
$$
\forall s, s' : \confSs{\wbSE{b}{S}{\neg b}}{s} \Rar{*} s' \iff  \confSs{\wbS{b}{S}}{s} \Rar{*} s')
$$
\begin{proof}
Dimostro le due implicazioni separatamente.

$$
\boxed{\implies}
$$

$$
\boxed{\impliedby}
$$
Assumo che $\confSs{\wbS{b}{S}}{s} \Rar{*} s'$. Applicando un unfolding ($\whileSOS$) ottengo:
$$
\confSs{\ifABC{b}{(S; \wbS{b}{S})}{\skipForFriends}}{s}
$$
In cui posso discriminare due casi in base a $\B{b}$:
\begin{itemize}
\item $\boxed{\B{b} = \semfalse}$
$$
\begin{array}{ll}
\confSs{\ifABC{b}{(S; \wbS{b}{S})}{\skipForFriends}}{s} & \ifffSOS\\
\confSs{\skipForFriends}{s} & \skipSOS\\
s & \\
\end{array}
$$
Da cui segue che $\boxed{s' \equiv{ } s}$
$$
\begin{array}{ll}
\confSs{\wbSE{b}{S}{\neg b}}{s} \Rar{ } & \exitifSOS\\
\confSs{\ifABC{b}{(S; (\ifABC{\neg b}{\skipForFriends}{(\wbSE{b}{S}{\neg b})}))}{\skipForFriends}}{s} \Rar{} & \ifffSOS \\
\confSs{\skipForFriends}{s} \Rar{}& \skipSOS\\
s & \\
\end{array}
$$
\item $\boxed{\B{b} = \semtrue}$
$$
\begin{array}{ll}
\confSs{\wbS{b}{S}}{s} \Rar{*} & \\ 
\confSs{\ifABC{b}{(S; \wbS{b}{S})}{\skipForFriends}}{s} \Rar{} & \ifttSOS\\
\confSs{S; \wbS{b}{S}}{s} & \\
\end{array}
$$	
Siccome le regole SOS sono deterministiche ottengo che
$$
\confSs{S; \wbS{b}{S}}{s} \Rar{*} s'
$$
e quindi per il lemma di decomposizione $\exists s''$:
\begin{enumerate}
\item 
$\confSs{S}{s} \Rar{*} s''$
\label{hw6:FattoA}
\item $\confSs{\wbS{b}{S}}{s''} \Rar{*} s' $
\label{hw6:FattoB}
\end{enumerate}
Inizio anche a sviluppare 
$$
\begin{array}{ll}
	\confSs{\wbSE{b}{S}{\neg b}}{s} \Rar{ } & \exitifSOS\\
	\confSs{\ifABC{b}{(S; (\ifABC{\neg b}{\skipForFriends}{(\wbSE{b}{S}{\neg b})}))}{\skipForFriends}}{s} \Rar{} & \ifttSOS \\
	\confSs{S; (\ifABC{\neg b}{\skipForFriends}{(\wbSE{b}{S}{\neg b})})}{s} & \\
\end{array}
$$
Usando il fatto \ref{hw6:FattoA}  e il lemma di composizione ottengo che:
$$
\begin{array}{ll}
\confSs{S; (\ifABC{\neg b}{\skipForFriends}{(\wbSE{b}{S}{b'})})}{s} \Rar{*} & \\
\confSs{\ifABC{\neg b}{\skipForFriends}{(\wbSE{b}{S}{b'})}}{s''} \Rar{}
\end{array}
$$
Ora invece di procedere come di consueto distinguo ulteriori due casi:
\begin{itemize}
\item $\boxed{\B{b}'' = \semfalse}$
Applicando questa ipotesi al fatto $\ref{hw6:FattoB}$ ottengo che $\boxed{s'' \equiv s'}$.
al contempo applicando questa ipotesi ottengo anche:
$$
\begin{array}{ll}
	\confSs{S; (\ifABC{\neg b}{\skipForFriends}{(\wbSE{b}{S}{b'})})}{s} \Rar{*} & \\
	\confSs{\ifABC{\neg b}{\skipForFriends}{(\wbSE{b}{S}{b'})}}{s''} \Rar{} & \ifttSOS \\
	\confSs{\skipForFriends}{s''} \Rar{ } s'' & \\
\end{array}
$$
quindi ho dimostrato che $\wbSE{b}{S}{\neg b} \Rar{*} s'$ perchè $s' \equiv s''$.
\item $\boxed{\B{b}'' = \semtrue}$	
In questo caso ottengo che
$$
\confSs{\wbS{b}{S}}{s''} \Rar{*} s'
$$
e a destra ottengo
$$
\confSs{\wbSE{b}{S}{\neg b}}{s''}
$$
ma quindi grazie all'ipotesi induttiva(???) ottengo che
$$
\confSs{\wbSE{b}{S}{\neg b}}{s''} \Rar{*} s'
$$
\end{itemize}
\end{itemize}
\end{proof}
\end{enumerate}
}