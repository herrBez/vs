\exercise{2.23}
{
  Mostrare che i seguenti \textit{statement} del linguaggio \textbf{While} sono
  semanticamente equivalenti:
  \begin{enumerate}[label=\alph*)]
    \item ``$S$; skip'' and ``$S$''
    \item ``while $b$ do $S$'' and
      ``if $b$ then ($S$; while $b$ do $S$) else skip''
    \item ``$S_1$; ($S_2$; $S_3$)'' and ``($S_1$; $S_2$); $S_3$''
  \end{enumerate}
}
{
  Di seguito vengono esposte le varie equivalenze tra gli statement sopra
  elencati:

  \begin{enumerate}[label=\alph*)]
    \item \boxer{``S; skip$''$ and ``S$''$}.

    \begin{itemize}
      \item $(\Rightarrow)$ Ipotizziamo che
      $\confSs{S;\mbox{ skip}}{s} \Rar{*} \gamma$. Per il lemma di
      decomposizione, esiste uno stato intermedio $s'$ t.c.
      $\confSs{S}{s} \Rar{*} s'$ e $\confSs{\mbox{skip}}{s'} \Rar{*} \gamma$.
      Per la regola SOS $[skip_{SOS}]$ si ha che $s' = \gamma$ (e quindi
      $\confSs{S}{s} \Rar{*} \gamma$).

      \item $(\Leftarrow)$ Ipotizziamo che $\confSs{S}{s} \Rar{*} \gamma$. Ciò
      significa che
      $\confSs{S; \mbox{skip}}{s} \Rar{*} \confSs{\mbox{skip}}{\gamma}$. \\
      Ma, per la regola $[skip_{SOS}]$,
      $\confSs{\mbox{skip}}{\gamma} \Rar{} \gamma$. Perciò
      $\confSs{S; skip}{s} \Rar{*} \gamma$.
    \end{itemize}


    \item \boxer{``while \ b \ do \ S$''$ \ and
      \ ``if  \ b \ then \ (S; \ while \ b \ do \ S) \ else \ skip$''$}. 
      Il secondo statement è
      l'equivalente del primo in semantica operazionale, perciò entrambi

    \item \boxer{``S_1; \ (S_2; \ S_3)$''$ \ and \ ``(S_1; \ S_2); \ S_3$''$}.
  \end{enumerate}
}
