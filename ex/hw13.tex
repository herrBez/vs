\newcommand{\powerset}[1]{\raisebox{.15\baselineskip}{\Large\ensuremath{\wp}}(#1)}

\exercise{Esercizio 13}
{
  Sia \setRel{P}{\leq} un poset con bottom $\bot$. Un sottoinsieme
  $S \subseteq P$ è \textit{down-closed} se
  $$
  \forall{x\in S}.\forall{y}\in P.(y \leq x \implies y \in S).
  $$
  Si osservi, in particolare, che $\emptyset$ è down-closed. Sia (per
  definizione) $dc(P) = \{S \subseteq P | S \text{è down-closed}\}$.
  \begin{enumerate}
    \item Provare che \setRel{dc(P)}{\subseteq} è un CPO. \\
  \noindent Si consideri ora il powerset CPO \setRel{\powerset{P}}{\subseteq} e
    la funzione $f : \powerset{P} \to dc(P)$ definita come segue:
    $$
    f(X) = \{p \in P | \exists{x}\in X.p \leq x\}.
    $$
    \item Provare che $f$ è ben definita, cioè che per ogni
      $X \in \powerset{P}$, $f(X) \in dc(P)$.
    \item Provare che $f$ è continua.
  \end{enumerate}
}
{}

\begin{proof}

\end{proof}

\undef{\powerset}
