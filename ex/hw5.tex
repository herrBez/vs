\addtocontents{toc}{\protect\setcounter{tocdepth}{1}}
\newcommand{\stmm}{\stm$^{-}$}

\exercise{Exercise 5}
{Sia \stmm{} un sottolinguaggio di \stm{} in cui le assegnazioni sono state
rimosse:
\begin{align*}
\mbox{\stmm} \ni{} S \Coloneqq{}  \skipForFriends\ | \ 
                                  S_1; S_2\ | \ 
                                  \ifABC{b}{S_1}{S_2}\ | \ 
                                  \wbS{b}{S}
\end{align*}
Provare che $\forall{S} \in$ \stmm{} e $\forall{s} \in$ \states, o
$\confSs{S}{s} \Rar{*} s$ oppure $\confSs{S}{s}$ non termina.
}
{}
Dopo aver fissato uno stato $s \in$ \states, è possibile procedere per
induzione strutturale sullo statement $S$.

\subsection{Caso base}

\begin{itemize}
  \item $S \equiv{} \skipForFriends$. In questo caso è facile mostrare che
    $\confSs{S}{s} \Rar{*} s$:
\begin{align*}
\confSs{\skipForFriends}{s} \Rar{} \skipSOS \ \ \ s
\end{align*}
\end{itemize}

\subsection{Passi induttivi}

\begin{itemize}
  \item Ipotesi Induttiva: $i \in \{1,2\}.S_i \in$ \stmm.
$\confSs{S_i}{s} 
\Rar{*} s$ oppure $\confSs{S_i}{s}$ non termina.
  \item $S \equiv{} S_1; S_2$. Per ipotesi induttiva, sappiamo che $S_1$ o
  termina in $s$ oppure non termina affatto:
  \begin{itemize}
    \item Se $\confSs{S_1}{s} \Rar{*} s$, allora per il lemma di composizione
      possiamo dire che $\confSs{S_1;S_2}{s} \Rar{*} \confSs{S_2}{s}$. Siccome
      l'ipotesi induttiva vale anche su $S_2$ in quanto sottotermine, si ha che
      $\confSs{S_2}{s} \Rar{*} s$ oppure $S_2$ ha lunghezza di derivazione
      infinita.
    \item Se $S_1$ non termina, allora l'intero termine $S$ avrà una
      derivazione di lunghezza infinita.
  \end{itemize}

  \item $S \equiv{} \ifABC{b}{S_1}{S_2}$. Per le regole $\ifffSOS{}$ e
    $\ifttSOS{}$, lo stato della memoria $s$ non viene modificato dopo la
    valutazione della guardia $b$. L'analisi si divide quindi in due sottocasi,
    a seconda che \B{b} sia valutato a \semtrue{} o \semfalse{}:
    \begin{itemize}
      \item $\B{b} = \semtrue{}$. La configurazione iniziale
        $\confSs{\ifABC{b}{S_1}{S_2}}{s}$ evolve in
        $\confSs{S_1}{s}$. Tuttavia, per ipotesi induttiva,
        $\confSs{S_1}{s} \Rar{*} s$ oppure $\confSs{S_1}{s}$ non termina.
      \item $\B{b} = \semfalse{}$. La configurazione iniziale
        $\confSs{\ifABC{b}{S_1}{S_2}}{s}$ evolve in
        $\confSs{S_2}{s}$. Tuttavia, per ipotesi induttiva,
        $\confSs{S_2}{s} \Rar{*} s$ oppure $\confSs{S_2}{s}$ non termina.
    \end{itemize}

  \item $S \equiv{} \wbS{b}{S_1}$ è banalmente verificato poichè, dopo un passo
    di valutazione $\whileSOS$, tale termine si riduce ad un termine condizionale
    $\ifABC{b}{(S_1;\wbS{b}{S_1})}{\skipForFriends}$. Questo statement equivale
    allo statement dimostrato al punto precendente, quindi anche questo caso è 
    verificato.
\end{itemize}

\undef{\stmm}
\addtocontents{toc}{\protect\setcounter{tocdepth}{2}}
