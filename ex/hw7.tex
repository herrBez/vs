\exercise{Esercizio 7}
{
  Si provi o si confuti la seguente equivalenza semantica:
$$
\wbS{b}{S} \eqSOS \wbS{b}{(S;\wbS{b}{S})}
$$
}
{
È necessario dimostrare che
$$
\forall s,s' \in \states. \Big(\confSs{\wbS{b}{S}}{s} \Rar{n_1} s'
  \iff
\confSs{\wbS{b}{(S;\wbS{b}{S})}}{s} \Rar{n_2} s' \Big)
$$
\begin{proof}
La prova si svolge per induzione sulla lunghezza della derivazione
(rispettivamente per $n_1$ e $n_2$ a seconda del verso della doppia
implicazione che si va a considerare).

Il caso $n = 0$ non viene mostrato per esteso in entrambi i versi in quanto è
vacuamente vero. Di seguito verranno dunque visti entrambi i versi delle
implicazioni per $n > 0$.
$$
\boxed{\implies}
$$
Si assuma che $\confSs{\wbS{b}{S}}{s} \Rar{*} s'$.
Per la regola $\whileSOS$ so che:
$$
\confSs{\wbS{b}{S}}{s} \Rar{} \confSs{\ifABC{b}{(S; \wbS{b}{S})}{\skipForFriends}}{s} \Rar{*} s'
$$
Si discriminino due casi in base alla valutazione della guardia $b$ nello
stato $s$:
\begin{itemize}
	\item $\boxed{\B{b} = \semfalse}$
	$$
	\begin{array}{lr}
	\confSs{\wbS{b}{S}}{s} \Rar{ } & \whileSOS \\
	\confSs{\ifABC{b}{(S; \wbS{b}{S})}{\skipForFriends}}{s} \Rar{ }& \ifffSOS \\
	\confSs{\skipForFriends}{s} \Rar{} & \skipSOS\\
	s
	\end{array}
	$$
	Quindi si ottiene che $s \equiv{} s'$.
	
	$$
	\begin{array}{ll}
	\confSs{\wbS{b}{(S;\wbS{b}{S})}}{s} \Rar{ } & \whileSOS\\	
	\confSs{\ifABC{b}{(S; \wbS{b}{S}); (\wbS{b}{(S;\wbS{b}{S})})}{\skipForFriends}}{s} \Rar{ } & \ifffSOS \\
	\confSs{\skipForFriends}{s} \Rar{ } & \skipSOS\\
	s
	\end{array}
	$$
	Di conseguenza quando $\B{b} = \semfalse$ i due
	statement sono equivalenti.
	
	\item $\boxed{\B{b} = \semtrue}$
	$$
	\begin{array}{lr}
	\confSs{\wbS{b}{S}}{s} \Rar{ }  & \whileSOS \\
	\confSs{\ifABC{b}{(S;\wbS{b}{S})}{\skipForFriends}}{s} \Rar{ } & \ifttSOS\\
	\confSs{S; \wbS{b}{S}}{s} \Rar{*} & (\text{Assunzione + Determinismo})\\
	s'
	\end{array}
	$$
	Per il lemma di decomposizione applicato a
	\begin{equation}
	\confSs{S; \wbS{b}{S}}{s} \Rar{*} s'
		\label{eq:hw7:imp}
	\end{equation}
	si ottiene che $\exists s'' \in \states$ tale che: 
	\begin{enumerate}[label=(\alph*)]
		\item $\confSs{S}{s} \Rar{*} s''$
		\item $\confSs{\wbS{b}{S}}{s''} \Rar{*} s'$
		\label{hw7:item:wbSbSs'':s'}
	\end{enumerate}
	Da \ref{hw7:item:wbSbSs'':s'} si ha, per l'esercizio 2, che:
	\begin{enumerate}[label=(\Roman*)]
		\item $\B{b}' = \semfalse$
		\label{hw7:Bbs':ff}
	\end{enumerate}
	% HERE
	Per ipotesi $\B{b} = \semtrue$, quindi si ha
	$$
	\begin{array}{lr}
	\confSs{\wbS{b}{(S; \wbS{b}{S})}}{s} \Rar{ } & \whileSOS\\
	\confSs{\ifABC{b}{(S; \wbS{b}{S}); \wbS{b}{(S; \wbS{b}{S})}}{\skipForFriends}}{s} \Rar{ }& \ifttSOS \\
	\confSs{(S; \wbS{b}{S}); \wbS{b}{(S; \wbS{b}{S})}}{s} &
	\end{array}
	$$
	Per il lemma di composizione applicato a (\ref{eq:hw7:imp}) si ricava che:
	\begin{equation}
	\confSs{(S; \wbS{b}{S}); (\wbS{b}{(S; \wbS{b}{S})})}{s} \Rar{*} \confSs{\wbS{b}{(S; \wbS{b}{S})}}{s'} 
	\label{hw7:c}
	\end{equation}
	da cui si ottiene, grazie a \ref{hw7:Bbs':ff}, che:
	$$
	\begin{array}{lr}
	\confSs{(S; \wbS{b}{S}); \wbS{b}{(S; \wbS{b}{S})}}{s} \Rar{*} & (\ref{hw7:c}) \\
	\confSs{\wbS{b}{(S; \wbS{b}{S})}}{s'} \Rar{} & \whileSOS\\
	\confSs{\ifABC{b}{(S; \wbS{b}{S}); \wbS{b}{(S; \wbS{b}{S})}}{\skipForFriends}}{s'} \Rar{}& \ifffSOS\\
	\confSs{\skipForFriends}{s'} \Rar{} & \skipSOS\\
	s' & \\
	\label{hw:arary}
	\end{array}
	$$
	Unendo questa derivazione al fatto che
	$\confSs{\wbS{b}{(S; \wbS{b}{S})}}{s'} \Rar{*} s'$, si ottiene
	proprio quanto ci si era prefissati di dimostrare, ovvero:
	$$
	\confSs{\wbS{b}{(S; \wbS{b}{S})}}{s} \Rar{*} s'
	$$
\end{itemize}

%%%%%%%%
%% SECOND PART
%%%%%%%%

$$
\boxed{\impliedby}
$$
Assumo che $\confSs{\wbS{b}{(S;\wbS{b}{S})}}{s} \Rar{*} s'$.
Applicando la regola $\whileSOS$ si ottiene:
$$\confSs{\wbS{b}{(S;\wbS{b}{S})}}{s}
	\Rar{ }
	\confSs{\ifABC{b}{((S; \wbS{b}{S});\wbS{b}{(S;\wbS{b}{S})})}{\skipForFriends}}{s}$$
Dipendentemente da $\B{b}$ vengono distinti due casi:
\begin{itemize}
	\item $\boxed{\B{b} = \semfalse}$: analogo a quanto dimostrato per il caso $\B{b} = \semfalse $ nella dimostrazione per $\boxed{\implies}$. 
	\item $\boxed{\B{b} = \semtrue}$:
	$$
	\begin{array}{lr}
	\confSs{\wbS{b}{(S; \wbS{b}{S})}}{s} \Rar{ }& \whileSOS \\
	\confSs{\ifABC{b}{((S; \wbS{b}{S});\wbS{b}{(S;\wbS{b}{S})})}{\skipForFriends}}{s} \Rar{ } & \ifttSOS \\
	\confSs{(S; \wbS{b}{S}); \wbS{b}{(S; \wbS{b}{S})}}{s}  \Rar{*} & \text{(Assunzione + Determinismo)} \\
	s'
	\end{array}
	$$
	Applicando il lemma di decomposizione, si ottiene che $\exists s'' \in \states$ tale che:
	\begin{enumerate}[label=(\arabic*)]
		\item $\confSs{S; \wbS{b}{S}}{s} \Rar{*} s''$
		\label{hw7:a}
		\item $\confSs{\wbS{b}{(S; \wbS{b}{S}})}{s''} \Rar{*} s'$
		\label{hw7:b}
	\end{enumerate}
	Inoltre è possibile applicare lo stesso lemma a \ref{hw7:a} per ottenere che
	$\exists s''' \in \states$ tale che:
	\begin{enumerate}[label=(\Alph*)]
		\item $\confSs{S}{s} \Rar{*} s'''$
		\item $\confSs{\wbS{b}{S}}{s'''} \Rar{*}s''$
		\label{hw7:justAnotherLabel}
	\end{enumerate}
	da \ref{hw7:justAnotherLabel} per l'esercizio 2 segue che:
	\begin{enumerate}[label=(\Roman*)]
		\item $\B{b}'' = \semfalse$
		\label{hw7:Bbs'':ff}
	\end{enumerate}
	Grazie a \ref{hw7:Bbs'':ff} si può concludere a partire da \ref{hw7:b} che:
	$$
	\begin{array}{ll}
		\confSs{\wbS{b}{(S; \wbS{b}{S})}}{s''} \Rar{ }& \whileSOS \\
		\confSs{\ifABC{b}{((S; \wbS{b}{S}); \wbS{b}{(S; \wbS{b}{S})})}{\skipForFriends}}{s''} \Rar{ } & \ifffSOS \\
		\confSs{\skipForFriends}{s''} \Rar{ }& \skipSOS\\
		s''
	\end{array}
	$$
	Da cui segue, per il determinismo delle regole SOS, che $s'' \equiv s'$.
	Considerando $\wbS{b}{S}$, grazie al fatto che
	$\B{b} = \semtrue$, è possibile derivare:
	$$
	\begin{array}{lr}
	\confSs{\wbS{b}{S}}{s} \Rar{ } & \whileSOS \\
	\confSs{\ifABC{b}{(S; \wbS{b}{S})}{\skipForFriends}}{s}  \Rar{ } & \ifttSOS \\
	\confSs{(S; \wbS{b}{S})}{s} & \\
	\end{array}
	$$
	Da cui grazie a \ref{hw7:a} e al fatto che $s'' \equiv s'$ si ha che
	$\confSs{(S; \wbS{b}{S})}{s} \Rar{*} s'$, quindi vale 
        $\confSs{\wbS{b}{S}}{s} \Rar{*} s'$.

\end{itemize}

\end{proof}


}
