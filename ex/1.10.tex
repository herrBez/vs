\exercise{1.10}
{Dimostrare che la tabella 1.2 definisce una funzione totale 
$\Bcal: \bexp \rightarrow (\state \rightarrow \T)$}
{
\begin{itemize}

  \item (base)
    \begin{itemize}
      
      \item $\boxed{b \equiv \syntrue \in \bexp}$ \\
            Banalmente vera: $\forall s \in \states$ la funzione $\Bcal$ mappa 
un unico elemento booleano sintattico ($\syntrue$) in un unico elemento 
booleano semantico ($\semtrue$). Quindi $\B{\syntrue}$ è una funzione totale 
perchè definita $\forall \syntrue \in \bexp$
      
      \item $\boxed{b \equiv \synfalse \in \bexp}$ \\
            La dimostrazione è identica al punto precedente sostituendo:
            \begin{enumerate}[label=(\alph*)] 
              \item $\syntrue$ con $\synfalse$
              \item $\semtrue$ con $\semfalse$
            \end{enumerate} 
 
    \end{itemize}

  \item (induttivo)
    \begin{itemize}
   
      \item $\boxed{b \equiv a_1 = a_2.a_1,a_2 \in \aexp}$ \\

      \item $\boxed{b \equiv a_1 \leq a_2.a_1,a_2 \in \aexp}$ \\

      \item $\boxed{b' \equiv \lnot b \in \bexp}$ \\

      \item $\boxed{b \equiv b_1 \land b_2.b_1,b_2 \in \bexp}$ \\
   
    \end{itemize}

\end{itemize}  
}
