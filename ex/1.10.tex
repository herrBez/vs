\exercise{1.10}
{Dimostrare che la tabella 1.2 definisce una funzione totale 
$\Bcal: \bexp \rightarrow (\state \rightarrow \T)$}
{
\begin{itemize}

  \item (base)
    \begin{itemize}
      
      \item $\boxed{b \equiv \syntrue \in \bexp}$ \\
            Banalmente vera: $\forall s \in \states$ la funzione $\Bcal$ mappa 
un unico elemento sintattico booleano ($\syntrue$) in un unico elemento 
semantico booleano ($\semtrue$). Quindi $\B{\syntrue}$ è una funzione totale 
perchè definita $\forall \syntrue \in \bexp$
      
      \item $\boxed{b \equiv \synfalse \in \bexp}$ \\
            La dimostrazione è identica al punto precedente sostituendo:
            \begin{enumerate}[label=(\alph*)] 
              \item $\syntrue$ con $\synfalse$
              \item $\semtrue$ con $\semfalse$
            \end{enumerate} 
   
      \item $\boxed{b \equiv a_1 = a_2.a_1,a_2 \in \aexp}$ \\
        \begin{table}[h!]
          \begin{center}
          \begin{tabular}{| l | l | p{0.4\linewidth} |}
            \hline
              $\A{a_1} = \A{a_2}$ &
              $\B{a_1 = a_2} = \A{a_1} = \A{a_2}$ &
              Definizione di $\B{a_1 = a_2}$ 
              \\ &
              $= \mathbf{v_1} = \mathbf{v_2}$ &
              $\A{a_1}$ e $\A{a_2}$: funzioni totali che restituiscono valori interi (dimostrato in 1.8) 
              \\ & 
              $= \semtrue$ & 
              Applicazione dell'operatore semantico = \\
              & & \\
              \hline
              & & \\ 
              $\A{a_1} \not = \A{a_2}$ &
              $\B{a_1 = a_2}  = \A{a_1} = \A{a_2}$ &
              Definizione di $\B{a_1 = a_2}$   
              \\ &
              $= \mathbf{v_1} = \mathbf{v_2}$ &
              $\A{a_1}$ e $\A{a_2}$: funzioni totali che restituiscono valori interi (dimostrato in 1.8)  
              \\ &              
              $= \semfalse$ & 
              Applicazione dell'operatore semantico = \\
              & & \\ 
            \hline
          \end{tabular}
          \end{center}
        \end{table}

      \item $\boxed{b \equiv a_1 \leq a_2.a_1,a_2 \in \aexp}$ \\
        \begin{table}[h!]
          \begin{center}
          \begin{tabular}{| l | l | p{0.4\linewidth} |}
            \hline
              $\A{a_1} \leq \A{a_2}$ &
              $\B{a_1 \leq a_2} = \A{a_1} \leq \A{a_2}$ &
              Definizione di $\B{a_1 \leq  a_2}$ 
              \\ &
              $= \mathbf{v_1} \leq \mathbf{v_2}$ &
              $\A{a_1}$ e $\A{a_2}$: funzioni totali che restituiscono valori interi (dimostrato in 1.8)  
              \\ &
              $= \semtrue$ &
              Applicazione dell'operatore semantico $\leq$ \\ 
              & & \\
              \hline
              & & \\ 
              $\A{a_1} \not \leq \A{a_2}$ &
              $\B{a_1 \leq a_2} = \A{a_1} \leq \A{a_2}$ & 
              Definizione di $\B{a_1 \leq a_2}$   
              \\ &
              $= \mathbf{v_1} \leq \mathbf{v_2}$ &
              $\A{a_1}$ e $\A{a_2}$: funzioni totali che restituiscono valori interi (dimostrato in 1.8) 
              \\ &              
              $= \semfalse$ & 
              Applicazione dell'operatore semantico $\leq$ \\
              & & \\ 
            \hline
          \end{tabular}
          \end{center}
        \end{table}

    \end{itemize}

  \item (induttivo)
    \begin{itemize}
      \item $\boxed{b' \equiv \lnot b \in \bexp}$ \\
        \begin{table}[h!]
          \begin{center}
          \begin{tabular}{| l | l | p{0.4\linewidth} |}
            \hline
              $\B{b} = \semfalse$ &
              $\B{\lnot b} = \B{b} = \semfalse$ &
              Definizione di $\B{\lnot b}$ 
              \\ &
              $= \mathbf{v} = \semfalse$ &
              Ipotesi induttiva applicata a $\B{b}$ 
              \\ & 
              $= \semtrue$ & 
              Applicazione dell'operatore semantico = \\
              & & \\
              \hline
              & & \\ 
              $\B{b} = \semtrue$ &
              $\B{\lnot b}  = \B{b} = \semfalse$ &
              Definizione di $\B{\lnot b}$   
              \\ &
              $= \mathbf{v} = \semfalse$ &
              Ipotesi induttiva applicata a $\B{b}$ 
              \\ &              
              $= \semfalse$ & 
              Applicazione dell'operatore semantico = \\
              & & \\ 
            \hline
          \end{tabular}
          \end{center}
        \end{table}
\newpage
      \item $\boxed{b \equiv b_1 \land b_2.b_1,b_2 \in \bexp}$ \\
         \begin{table}[h!]
          \begin{center}
          \begin{tabular}{| l | l | p{0.4\linewidth} |}
            \hline
              $\B{b_1} = \semtrue \ and \ \B{b_2} = \semtrue$ &
              $\B{b_1 \land b_2} = \B{b_1} \ and \ \B{b_2}$ &
              Definizione di $\B{b_1 \land b_2}$ 
              \\ &
              $= \mathbf{v_1} \ and \ \mathbf{v_2}$ &
              Ipotesi induttiva applicata a $\B{b_1} \ $ e $ \ \B{b_2}$ 
              \\ & 
              $= \semtrue$ & 
              Applicazione dell'operatore semantico $and$ \\
              & & \\
              \hline
              & & \\ 
              $\B{b_1} = \semfalse \ or \ \B{b_2} = \semfalse$ &
              $\B{b_1 \land b_2} = \B{b_1} \ and \ \B{b_2}$ &
              Definizione di $\B{b_1 \land b_2}$ 
              \\ &
              $= \mathbf{v_1} \ and \ \mathbf{v_2}$ &
              Ipotesi induttiva applicata a $\B{b_1} \ $ e $ \ \B{b_2}$ 
              \\ & 
              $= \semfalse$ & 
              Applicazione dell'operatore semantico $and$ \\
              & & \\
            \hline
          \end{tabular}
          \end{center}
        \end{table}  
    \end{itemize}

\end{itemize}
Per i punti precedenti $\Bcal: \bexp \rightarrow (\state \rightarrow \T)$ vale $\forall b \in \bexp, \forall s \in \state$ \\
$\Bcal$ è quindi una funzione definita per ogni input del suo dominio, ovvero una funzione totale
\begin{flushright}
$\Box$
\end{flushright}  
}
