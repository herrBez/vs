\exercise{Esercizio 12}
{Prove that
$$
	 \wbS{b}{(S; \wbS{b}{S})} \eqDS \wbS{b}{S}
$$
}
{
Per dimostrare che 
$$
	 \wbS{b}{(S; \wbS{b}{S})} \eqDS \wbS{b}{S}
$$	
è necessario dimostrare che
$$
\dsCtxt{\wbS{b}{(S; \wbS{b}{S})}} = \dsCtxt{\wbS{b}{S}}
$$
Il termine sulla destra è per quanto visto nei precedenti esercizi:
$$
\dsCtxt{\wbS{b}{S}} =
\fixp{F} \text{ dove } F = \lambda g. \cond{b}{g \circ \dsCtxt{S}}{id}
$$
mentre quello sulla sinistra è:
$$
\dsCtxt{\wbS{b}{(S; \wbS{b}{S}})} = \fixp{G} \text{ dove }
$$
dove
$$
G = \lambda g. \cond{b}{g \circ \fixp{F} \circ \dsCtxt{S}}{id}
$$

Si noti che in questo caso non è possibile dimostrare che $\fixp{F} = \fixp{G}$
a partire da $F = G$ perchè $F$ e $G$ non sono banalmente equivalenti.

\begin{proof}
Per dimostrare che $\fixp{F} = \fixp{G}$ sono equivalenti, le due direzioni di
prova vengono considerate separatamente:
\begin{itemize}
	\item $\boxed{\fixp{F} \sqsubseteq \fixp{G}}$
	Si tenta di usare il \FPIL; infatti, dimostrando che
	$$
	F(\fixp{G}) \sqsubseteq \fixp{G}
	$$
	si otterrebbe proprio il risultato cercato.
  Possiamo applicare $\FPIL$ perchè:
  \begin{itemize}
   \item $F$ è una funzione totale continua su CPO 
($\states \hookrightarrow \states, \sqsubseteq$) perchè $cond$ è continua
   \item $\fixp{G}$ è un funzionale ed ha tipo 
$$
((\states \hookrightarrow \states) \rightarrow 
(\states \hookrightarrow \states)) \rightarrow
(\states \hookrightarrow \states)
$$ 
quindi $\fixp{G} \in 
(\states \hookrightarrow \states)$.
  \end{itemize}
  Si proceda a sviluppare $F(\fixp{G})$:
  $$
  F(\fixp{G})\ s = \caseFun{\fixp{G} \circ \dsCtxt{S}\ s}
                        {\B{b} = \semtrue}
                        {\idDS\ s = s}
                        {\B{b} = \semfalse}
  $$
  Sviluppando anche $\fixp{G}$, si ottiene:
  $$
  \fixp{G}\ s =
  G(\fixp{G})\ s = \caseFun{(\fixp{G} \circ \fixp{F} \circ \dsCtxt{S})\ s}
                        {\B{b} = \semtrue}
                        {\idDS\ s = s}
                        {\B{b} = \semfalse}
  $$
  Si noti che, nel caso in cui $\B{b} = \semfalse$, si ha una perfetta
  equivalenza tra i due termini.
  Di conseguenza, è opportuno procedere la dimostrazione andando ad evolvere i
  casi $\B{b} = \semtrue$. Per il termine sulla sinistra si ha:
  $$
  (\fixp{G} \circ \dsCtxt{S})\ s
      = \caseFun{\fixp{G}\ s'}
                {\dsCtxt{S}\ s = s'}
                {\undefDS}
                {\dsCtxt{S}\ s = \undefDS}
  $$
  E per il lato destro:
  $$
  (\fixp{G} \circ \fixp{F} \circ \dsCtxt{S})\ s
      = \caseFun{(\fixp{G} \circ \fixp{F})\ s'}
                {\dsCtxt{S}\ s = s'}
                {\undefDS}
                {\dsCtxt{S}\ s = \undefDS}
  $$
  Come in precedenza, uno dei due casi è banalmente verificabile (ovvero quello
  in cui $\dsCtxt{S}\ s = \undefDS$). Si procede andando quindi ad evolvere il
  termine quando $\dsCtxt{S}\ s = s'$.
  $$
  \fixp{G}\ s' =
  G(\fixp{G})\ s' = \caseFun{(\fixp{G} \circ \fixp{F} \circ \dsCtxt{S})\ s'}
                        {\B{b}' = \semtrue}
                        {\idDS\ s' = s'}
                        {\B{b}' = \semfalse}
  $$
  Invece, per il termine sulla destra:
  $$
  (\fixp{G} \circ \fixp{F})\ s' =
  (\fixp{G}) \circ F (\fixp{F})\ s' =
                \caseFun{(\fixp{G} \circ \fixp{F} \circ \dsCtxt{S})\ s'}
                        {\B{b}' = \semtrue}
                        {(\fixp{G}) \circ \idDS\ s' = (\fixp{G})\ s'}
                        {\B{b}' = \semfalse}
  $$
  In questo caso, è facile notare che per $\B{b}' = \semtrue$ la proprietà su
  cui si sta facendo induzione è verificata. Per il caso $\B{b}' = \semfalse$ è
  invece sufficiente analizzare con più attenzione il termine ottenuto a
  partire da ($\fixp{G} \circ \fixp{F})\ s'$:
  $$
  \begin{array}{lr}
  (\fixp{G})\ s' = & (\text{applicazione punto fisso}) \\
  (G (\fixp{G}))\ s' = & \B{b}' = \semfalse \\
  \idDS\ s' = & \text{definizione di \idDS} \\
  s'
  \end{array}
  $$
  Perciò si ha che anche il caso $\B{b}' = \semfalse$ è verificato.


	\item $\boxed{\fixp{F} \sqsupseteq \fixp{G}}$
	Si tenta di usare il \FPIL; infatti, dimostrando che
	$$
	G(\fixp{F}) \sqsubseteq \fixp{F}
	$$
	si otterrebbe proprio il risultato cercato.
        Possiamo applicare $\FPIL$ perchè:
        \begin{itemize}
         \item $G$ è una funzione totale continua su CPO 
          ($\states \hookrightarrow \states, \sqsubseteq$) perchè $cond$ 
          è continua
         \item $\fixp{F}$ è un funzionale ed ha tipo 
          $$
          ((\states \hookrightarrow \states) \rightarrow 
          (\states \hookrightarrow \states)) \rightarrow
          (\states \hookrightarrow \states)
          $$ 
          quindi $\fixp{F} \in 
          (\states \hookrightarrow \states)$.
        \end{itemize}
        Innanzitutto si sviluppi $(G\ \fixp{F})$ su un stato $s$ qualsiasi:
        $$
        \begin{array}{ll}
          (G \ \fixp{F})\ s & = (\cond{b}{\fixp{F} \circ\  \fixp{F} \circ \
          \dsCtxt{S}}{id})\ s \\
          & \\
          & = \caseFun{(\fixp{F} \circ{} \fixp{F} \circ{} \dsCtxt{S})\ s}{\B{b} =
          \semtrue}{s}{\B{b} = \semfalse}
        \end{array}
        $$
        Nel caso $\B{b} = \semtrue$ è possibile distinguere due casi in base
        alla valutazione di $\dsCtxt{S}$ applicata allo stato $s$:
        $$
        \begin{array}{ll}
        \dots = 
        & \caseFun{(\fixp{F} \circ \fixp{F} \circ \dsCtxt{S})\ s}{\B{b} =
          \semtrue}{s}{\B{b} = \semfalse}  \\
        \\
        & = \left\lbrace 
          \begin{array}{ll}
            undef & \mbox{if } \B{b} = \semtrue \land (\dsCtxt{S}\  s) = undef  \\
            &  \\                  
            (\fixp{F} \circ \fixp{F})\ s' & \mbox{if } \B{b} = \semtrue \land (\dsCtxt{S}\ s) = s'  \\
                  & \\
            \idDS\ s = s & \B{b} = \semfalse\\
          \end{array}
          \right. \\
        \end{array}
        $$

        % TODO @Stefano: Vuoi scrivere gli altri due casi?
        % In sostanza sarebbe da mostrare l'evoluzione di (FIX F) andando a
        % distinguere solamente i casi B[[b]]s e \dsCtxt{S}\ s)
        Mentre l'equivalenza è banalmente verificata per i due casi $\undefDS$
        e $\idDS$, occorre studiare con attenzione il caso
        $(\fixp{F} \circ \fixp{F})\ s'$ (dato
        $\B{b} = \semtrue \land (\dsCtxt{S}\ s) = s'$):
        $$
        \begin{array}{ll}
        \fixp{F} (\fixp{F}\ s') =
        & \caseFun{\undefDS}
                  {\fixp{F}\ s' = \undefDS}
                  {\fixp{F}\ s''}
                  {\fixp{F}\ s' = s''}  \\
        \\
        \end{array}
        $$

        Notando che $\fixp{F}\ s'' = F(\fixp{F})\ s''$, per l'esercizio 9 si
        ha che $\B{b}'' = \semfalse$. Quindi:
        $$
        F(\fixp{F})\ s'' = \idDS{}\ s'' = s''
        $$

        Ma, sotto le stesse ipotesi, è possibile mostrare che
        $\fixp{F}\ s  =s''$.
        $$
        \begin{array}{lr}
        \fixp{F}\ s = & \text{def. punto fisso} \\
        F(\fixp{F}\ s) = & (\B{b} = \semtrue) \\
        (\fixp{F} \circ \dsCtxt{S})\ s = & (\dsCtxt{S} = s') \\
        \fixp{F} \ s' = & (\fixp{F}\ s' = s'') \\
        s''
        \end{array}
        $$
\end{itemize}

\end{proof}
\paragraph{Derivazione di G}
$$
\begin{array}{ll}
\dsCtxt{\wbS{b}{(S; \wbS{b}{S})}} & \\
\dsCtxt{\ifABC{b}{((S; \wbS{b}{S}); \wbS{b}{(S; \wbS{b}{S})})}{\skipForFriends}} & \\
\cond{b}{\dsCtxt{((S; \wbS{b}{S}); \wbS{b}{(S; \wbS{b}{S})})}}{\dsCtxt{\skipForFriends}}\\
\cond{b}{\dsCtxt{\wbS{b}{(S; \wbS{b}{S})}} \circ \dsCtxt{S; \wbS{b}{S}}}{id} & \\
\cond{b}{\dsCtxt{\wbS{b}{(S; \wbS{b}{S})}} \circ \dsCtxt{\wbS{b}{S}} \circ \dsCtxt{S}}{id} & \\
\cond{b}{\dsCtxt{\wbS{b}{(S; \wbS{b}{S})}} \circ \fixp{F} \circ \dsCtxt{S}}{id} & \\
\cond{b}{g \circ \fixp{F} \circ \dsCtxt{S}}{id} & \\
\end{array}
$$
quindi:
$$
G = \cond{b}{g \circ \fixp{F} \circ \dsCtxt{S}}{id}
$$
}
