\exercise{Esercizio 12}
{Prove that
$$
	 \wbS{b}{(S; \wbS{b}{S})} \eqDS \wbS{b}{S}
$$
}
{
Per dimostrare che 
$$
	 \wbS{b}{(S; \wbS{b}{S})} \eqDS \wbS{b}{S}
$$	
devo dimostrare che
$$
\dsCtxt{\wbS{b}{(S; \wbS{b}{S})}} = \dsCtxt{\wbS{b}{S}} 
$$
Il termine sulla destra è per quanto visto nei precedenti esercizi:
$$
\dsCtxt{\wbS{b}{S}} = \fixp{F} \text{ dove } F = \lambda g. \cond{b}{g \circ \dsCtxt{S}}{id}
$$
mentre quello sulla sinistra è:
$$
\dsCtxt{\wbS{b}{(S; \wbS{b}{S}})} = \fixp{G} \text{ dove } 
$$
dove 
$$
G = \lambda g. \cond{b}{g \circ \fixp{F} \circ \dsCtxt{S}}{id} 
$$

Noto che in questo caso non posso dimostrare che $\fixp{F} = \fixp{G}$ a partire da $F = G$ perchè
$F$ e $G$ non sono equivalenti.

\begin{proof}
Per dimostrare che $\fixp{F} = \fixp{G}$ sono equivalenti dimostro le due direzioni separatamente:
\begin{itemize}
	\item $\boxed{\fixp{F} \sqsubseteq \fixp{G}}$
	Cerco di sfruttare il \FPIL. Dimostrando che
	$$
	F(\fixp{G}) \sqsubseteq \fixp{G} 
	$$
	otterrei proprio il risultato cercato.
	\item $\boxed{\fixp{F} \sqsupseteq \fixp{G}}$
	Cerco di sfruttare il \FPIL. Dimostrando che
	$$
	G(\fixp{F}) \sqsubseteq \fixp{F} 
	$$
	otterrei proprio il risultato cercato.
        Inizio sviluppando $(G(\fixp{F}))$ su un stato $s$ qualsiasi:
        $$
        \begin{array}{ll}
          (G \ \fixp{F})\ s & = (\cond{b}{\fixp{F} \circ\  \fixp{F} \circ \
          \dsCtxt{S}}{id})\ s \\
          & \\
          & = \caseFun{(\fixp{F} \circ{} \fixp{F} \circ{} \dsCtxt{S})\ s}{\B{b} =
          \semtrue}{s}{\B{b} = \semfalse}
        \end{array}
        $$
        Nel caso $\B{b} = \semtrue$ posso distinguere due casi in base
        alla valutazione di $\dsCtxt{S}$ applicata allo stato $s$:
        $$
        \begin{array}{ll}
        \dots = 
        & \caseFun{(\fixp{F} \circ \fixp{F} \circ \dsCtxt{S})\ s}{\B{b} =
          \semtrue}{s}{\B{b} = \semfalse}  \\
        \\
        & = \left\lbrace 
          \begin{array}{ll}
            undef & \mbox{if } \B{b} = \semtrue \land (\dsCtxt{S}\  s) = undef  \\
            &  \\                  
            (\fixp{F} \circ \fixp{F})\ s & \mbox{if } \B{b} = \semtrue \land (\dsCtxt{S}\ s) = s'  \\
                  & \\
            id & \B{b} = \semfalse\\
          \end{array}	
          \right. \\
        \end{array}
        $$
        Questa funzione assomiglia molto a $F(\fixp{F})$ ($\equiv \fixp{F}$) se non fosse per il
        caso:
        $$
        \begin{array}{l}
        (\fixp{F} \circ \ \fixp{F})\ s
        \end{array}
        $$
        Ora mostro che $(\fixp{F} \circ \ \fixp{F})\ s = (\fixp{F} s)$.
        Siccome:
        $$
        (\fixp{F}\ s) = \caseFun{s'}{\fixp{F}\ s = s'}{undef}{\fixp{F}\ s = undef}
        $$
        se $\fixp{F}\ s$ è definito ritorna uno stato $s'$, per cui per l'esercizio
        $9$ vale $\B{b}' = \semfalse$.
        
        Inoltre siccome $\fixp{F}$ è un punto fisso è equivalente a $F(\fixp{F})$ per
        definizione, quindi
        $$
        F(\fixp{F}) s' = (\cond{b}{\fixp{F} \circ \dsCtxt{S}}{id})\ s' = s'
        $$
        Quindi ho ottenuto che se, $(\fixp{F}\ s)$ è definito $(\fixp{F} \circ\ \fixp{F})\ s = (\fixp{F}\ s') =
        s' = \fixp{F}\ s$.

        Che era proprio ciò che si doveva dimostrare.
        
\end{itemize}

\end{proof}
\paragraph{Derivazione di G}
$$
\begin{array}{ll}
\dsCtxt{\wbS{b}{(S; \wbS{b}{S})}} & \\
\dsCtxt{\ifABC{b}{((S; \wbS{b}{S}); \wbS{b}{(S; \wbS{b}{S})})}{\skipForFriends}} & \\
\cond{b}{\dsCtxt{((S; \wbS{b}{S}); \wbS{b}{(S; \wbS{b}{S})})}}{\dsCtxt{\skipForFriends}}\\
\cond{b}{\dsCtxt{\wbS{b}{(S; \wbS{b}{S})}} \circ \dsCtxt{S; \wbS{b}{S}}}{id} & \\
\cond{b}{\dsCtxt{\wbS{b}{(S; \wbS{b}{S})}} \circ \dsCtxt{\wbS{b}{S}} \circ \dsCtxt{S}}{id} & \\
\cond{b}{\dsCtxt{\wbS{b}{(S; \wbS{b}{S})}} \circ \fixp{F} \circ \dsCtxt{S}}{id} & \\
\cond{b}{g \circ \fixp{F} \circ \dsCtxt{S}}{id} & \\
\end{array}
$$
quindi:
$$
G = \cond{b}{g \circ \fixp{F} \circ \dsCtxt{S}}{id}
$$
}
