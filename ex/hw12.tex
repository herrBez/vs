\exercise{Esercizio 12}
{Prove that
$$
	 \wbS{b}{(S; \wbS{b}{S})} \eqDS \wbS{b}{S}
$$
}
{
Per dimostrare che 
$$
	 \wbS{b}{(S; \wbS{b}{S})} \eqDS \wbS{b}{S}
$$	
devo dimostrare che
$$
\dsCtxt{\wbS{b}{(S; \wbS{b}{S})}} = \dsCtxt{\wbS{b}{S}} 
$$
Il termine sulla destra è per quanto visto nei precedenti esercizi:
$$
\dsCtxt{\wbS{b}{S}} = \fixp{F} \text{ dove } F = \lambda g. \cond{b}{g \circ \dsCtxt{S}}{id}
$$
mentre quello sulla sinistra è:
$$
\dsCtxt{\wbS{b}{(S; \wbS{b}{S}})} = \fixp{G} \text{ dove } 
$$
dove 
$$
G = \lambda g. \cond{b}{g \circ \fixp{F} \circ \dsCtxt{S}}{id} 
$$

Noto che in questo caso non posso dimostrare che $\fixp{F} = \fixp{G}$ a partire da $F = G$ perchè
$F$ e $G$ non sono equivalenti.

\begin{proof}
Per dimostrare che $\fixp{F} = \fixp{G}$ sono equivalenti dimostro le due direzioni separatamente:
\begin{itemize}
	\item $\boxed{\fixp{F} \sqsubseteq \fixp{G}}$
	Cerco di sfruttare il \FPIL. Dimostrando che
	$$
	F(\fixp{G}) \sqsubseteq \fixp{G} 
	$$
	otterrei proprio il risultato cercato.
	\item $\boxed{\fixp{F} \sqsupseteq \fixp{G}}$
	Cerco di sfruttare il \FPIL. Dimostrando che
	$$
	G(\fixp{F}) \sqsubseteq \fixp{F} 
	$$
	otterrei proprio il risultato cercato.
\end{itemize}

\end{proof}
\paragraph{Derivazione di G}
$$
\begin{array}{ll}
\dsCtxt{\wbS{b}{(S; \wbS{b}{S})}} & \\
\dsCtxt{\ifABC{b}{((S; \wbS{b}{S}); \wbS{b}{(S; \wbS{b}{S})})}{\skipForFriends}} & \\
\cond{b}{\dsCtxt{((S; \wbS{b}{S}); \wbS{b}{(S; \wbS{b}{S})})}}{\dsCtxt{\skipForFriends}}\\
\cond{b}{\dsCtxt{\wbS{b}{(S; \wbS{b}{S})}} \circ \dsCtxt{S; \wbS{b}{S}}}{id} & \\
\cond{b}{\dsCtxt{\wbS{b}{(S; \wbS{b}{S})}} \circ \dsCtxt{\wbS{b}{S}} \circ \dsCtxt{S}}{id} & \\
\cond{b}{\dsCtxt{\wbS{b}{(S; \wbS{b}{S})}} \circ \fixp{F} \circ \dsCtxt{S}}{id} & \\
\cond{b}{g \circ \fixp{F} \circ \dsCtxt{S}}{id} & \\
\end{array}
$$
quindi:
$$
G = \cond{b}{g \circ \fixp{F} \circ \dsCtxt{S}}{id}
$$
}