\exercise{Esercizio 9}
{
  Provare che se $\dsCtxt{\wbS{b}{S}} s = s'$, allora $\B{b}' = \semfalse$.
}
{
}

Sia $F$ il funzionale associato a $\wbS{b}{S}$. Si può dunque dire che
$$
\dsCtxt{\wbS{b}{S}} = \fixp{F}
$$
dove $F = \lambda{g}.\cond{b}{g \circ \dsCtxt{S}}{\idDS}$.

Siccome $\dsCtxt{\wbS{b}{S}} s = s'$, allora
$$
\exists{n}\in\mathbb{N}. (F^n \bot)\ s = s'
$$

La tesi $\B{b}' = \semfalse$\footnote{N.B. l'esercizio può essere anche
formalizzato nel seguente modo:
$\forall{s\in\states}.\dsCtxt{\wbS{b}{S}} s = s' \implies \B{b}' = \semfalse$} 
viene dunque provata per induzione su $n$.

$$
\boxer{Caso\ base}
$$

Per $n = 0$, si ha che $F^0 = \idDS$. Ciò vuol dire che
$(F^0 \bot) s = \undefDS$.

Ma allora l'ipotesi $\dsCtxt{\wbS{b}{S}} s = s'$ è falsa,
poichè non esiste alcun $s'$ che la soddisfi.

Il caso base è dunque vacuamente vero.

$$
\boxer{Passo\ induttivo}
$$

Si consideri il caso $F^{n+1}$:
\vspace{.5em}

\noindent $F^{n+1} =$ \\
\noindent $F(F^n) =$ \\
\noindent $\cond{b}{F^n \circ \dsCtxt{S}}{\idDS}$
\vspace{.5em}

In definitiva, bisogna considerare la valutazione di
$(F^{n+1} \bot) s = (\cond{b}{((F^n \bot) \circ \dsCtxt{S})}{\idDS})\ s$.

A questo punto si può dividere la prova del passo induttivo in due casi a
seconda della valutazione di \B{b}:
\begin{itemize}
  \item $\B{b} = \semfalse$ -- in questo caso lo stato $s$ raggiunto è uguale
    allo stato finale $s'$ poichè viene applicata la funzione identità \idDS{}.
    Perciò (banalmente) $\B{b} = \B{b}' = \semfalse$.
  \item $\B{b} = \semtrue$ -- in questo caso viene scelto il primo ramo
    condizionale e perciò viene valutato
    $((F^n \bot) \circ \dsCtxt{S})\ s$.
    \begin{itemize}
      \item $\dsCtxt{S} s = \undefDS$, l'implicazione è vera poichè l'ipotesi
        diventa falsa;
      \item $\exists{s''}\in\states.\dsCtxt{S} s = s''$, ma in questo caso
        $(F^n \bot)\ s''$ arriva (per ipotesi induttiva) ad uno stato $s'$ per
        cui vale $\B{b}' = \semfalse$.
    \end{itemize}
\end{itemize}
