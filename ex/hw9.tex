\exercise{Esercizio 9}
{
  Provare che se $\dsCtxt{\wbS{b}{S}} s = s'$, allora $\B{b}' = \semfalse$.
}
{
}

Sia $F$ il funzionale associato a $\wbS{b}{S}$. Si può dunque dire che
$$
\dsCtxt{\wbS{b}{S}} = FIX F
$$
, dove $F = \lambda{g}.\cond{\Bcal{b}}{g \circ \dsCtxt{S}}{\idDS}$.

Siccome $\dsCtxt{\wbS{b}{S}} s = s'$, allora
$$
\exists{n}\in\Ncal. F^n \bot = s'
$$

La tesi $\B{b}' = \semfalse$ viene dunque provata per induzione su $n$.

$$
\boxer{Caso base}
$$

Per $n = 0$, si ha che $F^0 = \idDS$. Ciò vuol dire che $F^n \bot = \bot$.

Ma se $F^n = \bot$, allora l'ipotesi $\dsCtxt{\wbS{b}{S}} s = s'$ è falsa,
poichè non esiste alcun $s'$ che la soddisfi.
