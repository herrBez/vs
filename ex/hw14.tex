\addtocontents{toc}{\protect\setcounter{tocdepth}{1}}
\newcommand{\lt}{\sqsubseteq}

\exercise{Exercise 14}
{Sia D un CPO e siano $f,g: D \to{} D$ delle funzioni continue. Provare che:
\begin{align*}
\fixp{f \circ g} = f(\fixp{g \circ f})
\end{align*}
}
{}
Dal momento che $f$ e $g$ sono funzioni continue su un CPO, allora:

\begin{itemize}
  \item $g \circ f$ e $f \circ g$ sono continue (per lemma 4.35);
  \item $\fixp{f \circ g}$ esiste per il teorema di Knaster-Tarski-Kleene;
  \item $\fixp{g \circ f}$ esiste per il teorema di Knaster-Tarski-Kleene.
\end{itemize}

Al fine di provare la tesi, sviluppiamo la dimostrazione spezzando l'uguaglianza
in due versi\footnote{viene supposto che la relazione di ordinamento sul CPO D
sia $\lt$}:

\begin{enumerate}[label=(\arabic*)]
  \item $\fixp{f \circ g} \lt f(\fixp{g \circ f})$
  \item $f(\fixp{g \circ f}) \lt \fixp{f \circ g}$
\end{enumerate}

\subsection{(1)}
La prima disequazione viene dimostrata tramite il \FPIL.
$$
f \circ g(f(\fixp{g \circ f})) \lt f(\fixp{g \circ f})
\implies
\fixp{f \circ g} \lt f(\fixp{g \circ f})
$$

È quindi sufficiente provare che:
\begin{align*}
f \circ g(f(\fixp{g \circ f})) \lt f(\fixp{g \circ f})
\end{align*}
\begin{align*}
f \circ g \circ f(\fixp{g \circ f}) \lt f(\fixp{g \circ f})
\end{align*}
\begin{align*}
f (g \circ f(\fixp{g \circ f})) \lt f(\fixp{g \circ f})
\end{align*}

Ma è sufficiente applicare il punto fisso a destra dell'equazione:
\begin{align*}
f (g \circ f(\fixp{g \circ f})) \lt f (g \circ f(\fixp{g \circ f}))
\end{align*} \qed

\subsection{(2)}

Applico una volta il punto fisso sulla parte destra dell'equazione; ottengo
quindi:

\begin{align*}
f(\fixp{g \circ f}) \lt f \circ g(\fixp{f \circ g})
\end{align*}
\begin{align*}
f(\fixp{g \circ f}) \lt f(g(\fixp{f \circ g}))
\end{align*}

Sapendo che $f$ è una funzione monotona (poichè continua), possiamo dire che:
$$
\fixp{g \circ f} \lt g(\fixp{f \circ g})
\Rightarrow
f(\fixp{g \circ f}) \lt f(g(\fixp{f \circ g}))
$$

L'obiettivo è quindi dimostrare che $\fixp{g \circ f} \lt g(\fixp{f \circ g})$,
così da dimostrare la tesi del punto corrente. Tuttavia, ora la disequazione è
in una forma per cui è possibile utilizzare il \FPIL.
$$
g \circ f(g(\fixp{f \circ g})) \lt g(\fixp{f \circ g})
\implies
\fixp{g \circ f} \lt g(\fixp{f \circ g})
$$

Perciò si dimostra l'ipotesi del \FPIL:

\begin{align*}
g \circ f(g(\fixp{f \circ g})) \lt g(\fixp{f \circ g})
\end{align*}
\begin{align*}
g \circ f(g(\fixp{f \circ g})) \lt g(f \circ g(\fixp{f \circ g}))
\end{align*}
\begin{align*}
g(f(g(\fixp{f \circ g}))) \lt g(f(g(\fixp{f \circ g})))
\end{align*}

Il che è banale. \qed

\let\lt\undefined
\addtocontents{toc}{\protect\setcounter{tocdepth}{2}}
