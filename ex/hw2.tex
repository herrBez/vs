\exercise{Esercizio 2}
{
 Si provi che se \confSs{\wbS{b}{S}}{s} \Rar{*} $s'$ allora
 $\B{b}' = \semfalse$.
}
{
Assumo che \confSs{\wbS{b}{S}}{s} $\Rar{n}$ $s'$ per un qualche $n$.

Allo statement di partenza posso applicare solo la regola \whileSOS
(essendo la semantica SOS deterministica), da cui ottengo:
$$
  \confSs{\ifABC{b}{S; \wbS{b}{S}}{\skipForFriends}}{s} 
$$
che dipendentemente dalla valutazione di $b$ può:
\begin{itemize}
  \item eseguire il ramo \texttt{else}, in cui dopo una 
    \skipForFriends{} terminerà 
  \item eseguire il ramo \texttt{if}, in cui eseguirà (almeno) 
    un'altra iterazione
\end{itemize}
\begin{proof}
Procedo per induzione sulla lunghezza della derivazione.

\begin{itemize}
  \item $\boxed{\text{Caso Base}: n = 0}$ L'ipotesi è vacuamente verificata
    poichè \confSs{\wbS{b}{S}}{s} $\not\Rar{0}$ $s'$.

  \item $\boxed{\text{Passo Induttivo}: n > 0}$
    \begin{itemize}
      \item Caso banale: $\boxed{\B{b} = \semfalse}$
        $$
        \begin{array}{lrr}
        \confSs{\wbS{b}{S}}{s} \Rar{} & \whileSOS & \\
        \confSs{\ifABC{b}{S; \wbS{b}{S}}{\skipForFriends}}{s}
            \Rar{} & \ifffSOS & \\
        \confSs{\skipForFriends}{s} \Rar{} & \skipSOS & \\
        s & & \\
        \end{array}
        $$
        Siccome $s = s'$ e $\B{b} = \B{b}' = \semfalse$ l'asserto è dimostrato.
        Resta quindi da valutare il caso $\boxed{\B{b} = \semtrue}$.
      \item Ipotesi Induttiva: \confSs{\wbS{b}{S}}{t} \Rar{n} $s'$ then
        $\B{b}' = \semfalse$, con $t\in\states$
    \end{itemize}
    $$
    \begin{array}{lrr}
    \confSs{\wbS{b}{S}}{s} \Rar{} & \whileSOS & \\
    \confSs{\ifABC{b}{S; \wbS{b}{S}}{\skipForFriends}}{s} \Rar{} & \ifttSOS & \\
    \confSs{S; \wbS{b}{S}}{s} \Rar{*} & (Determinismo + Assunzione) & \\
    s' & & \\
    \end{array}
    $$

   Per il lemma di decomposizione so che $\exists s''\in\states$:
   \begin{itemize}
   \item $\exists l\in\mathbb{N}. \confSs{S}{s} \Rar{l} s''$
   \item $\exists k\in\mathbb{N}. \confSs{\wbS{b}{S}}{s''} \Rar{k} s'$
   \end{itemize}
   siccome la derivazione di \confSs{\wbS{b}{S}}{s''} \Rar{k} $s'$ è più
   corta di quella di partenza (\confSs{\wbS{b}{S}}{s} \Rar{n}
   $s'$), posso applicare l'ipotesi induttiva e concludere che $\B{b}'$ deve
   necessariamente essere uguale a \semfalse.

  \end{itemize} 
 \end{proof}
}
