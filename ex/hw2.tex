\exercise{Esercizio 2}
         {
           Prove that if \confSs{\wbS{b}{S}}{s} \Rar{*} $s'$ then $\B{b}' = \semfalse$
         }
         {
           Assumo che \confSs{\wbS{b}{S}}{s} $\Rar{*}$ $s'$.

           Siccome la semantica SOS è deterministica allo Statement di
           partenza posso applicare solo la regola \whileSOS, da cui ottengo
           il seguente statement:
           \confSs{\ifABC{b}{S; \wbS{b}{S}}{\skipForFriends}}{s} 
           che dipendentemente dalla valutazione di $b$ può andare nel ramo
           \texttt{else}, in cui dopo una \skipForFriends{} terminerà o andrà nel ramo
           \texttt{if}, in cui eseguirà (almeno) un'altra iterazione.

           \begin{proof}
                        Procedo per induzione sulla lunghezza della derivazione.

           \begin{itemize}
             \item $\boxed{\text{Caso Base}: n = 3}$ ovvero 
           $\boxed{\B{b} = \semfalse}$ 
           
           \confSs{\wbS{b}{S}}{s} \Rar{\whileSOS}
           \confSs{\ifABC{b}{S; \wbS{b}{S}}{\skipForFriends}}{s} 
           \Rar{\ifffSOS} \confSs{\skipForFriends}{s} 
           \Rar{\skipSOS} $s \equiv s'$

           Siccome $s = s'$ e $\B{b} = \semfalse$ l'asserto è dimostrato.

           \item $\boxed{\text{Ipotesi Induttiva}} $ \confSs{\wbS{b}{S}}{s} \Rar{*} $s'$ then $\B{b}' = \semfalse$

           \item $\boxed{\text{Passo Induttivo}: n \rightarrow n + l,
             \text{ dove } \confSs{S}{s} \Rar{l} s' \text{, ovvero } \B{b} = \semtrue}$

             \confSs{\wbS{b}{S}}{s} \Rar{\whileSOS} \confSs{\ifABC{b}{S; \wbS{b}{S}}{\skipForFriends}}{s} 
             \Rar{\ifttSOS} \confSs{S; \wbS{b}{S}}{s} \Rar{*} $s'$

             Per il lemma di decomposizione so che $\exists s''$:
             \begin{itemize}
             \item \confSs{S}{s} \Rar{*} $s''$ (ovvero $\exists l:$
               \confSs{S}{s} \Rar{l} $s''$)
             \item \confSs{\wbS{b}{S}}{s''} \Rar{*} $s'$
             \end{itemize}
             siccome la derivazione di \confSs{\wbS{b}{S}}{s''} \Rar{*} $s'$ è più
             corta di quella di partenza (\confSs{\wbS{b}{S}}{s} \Rar{*} $s'$)
             di $l$ passi, posso applicare l'ipotesi
             induttiva e concludere che $\B{b}'$ deve neccessariamente essere
             uguale \semfalse.
             
            \end{itemize} 
           \end{proof}
         }
