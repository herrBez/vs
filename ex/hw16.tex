\exercise{Esercizio 16}
{
  Sia \setRel{D}{\leq} un CPO. Siano $f: D \to D$ e $g: D \to D$ due funzioni
  continue tali che $\forall{x}\in D. f(x) \leq{} g(x)$.
  \begin{enumerate}
    \item Provare che $\fixp{f} \leq{} \fixp{g}$.
    \item Si assuma anche che $\forall{x}\in D.x \leq{} f(x)$. Provare che
      $\fixp{f} \leq \fixp{g^2}$. (dove $g^2 \equiv{} g \circ g$)
  \end{enumerate}
}
{}
L'esercizio procede provando le due proprietà separatamente.

\begin{proof}
$$
\boxed{\fixp{f} \leq{} \fixp{g}}
$$

Dal momento che $f$ e $g$ sono continue su un CPO $D$, allora esistono i
rispettivi punti fissi \fixp{f} e \fixp{g}.

Tale disuguaglianza viene dimostrata applicando il \FPIL:
\begin{equation}
f(\fixp{g}) \leq \fixp{g}  \Rar{} \fixp{f} \leq{} \fixp{g}
\label{hw16-x}
\end{equation}
Applicando una volta il punto sulla parte destra di $f(\fixp{g}) \leq \fixp{g}$,
si ottiene:
$$
f(\fixp{g}) \leq g(\fixp{g})
$$

Che è vero poichè:
\begin{itemize}
  \item $\fixp{g} \in D$;
  \item $\forall{x}\in D. f(x) \leq{} g(x)$.
\end{itemize}

$$
\boxed{\fixp{f} \leq \fixp{g^2}}
$$

In questa parte dell'esercizio è possibile assumere che
$\forall{x}\in D.x \leq{} f(x)$.

Si procede dunque provando ad applicare il \FPIL{}; per provare
$\fixp{f} \leq \fixp{g^2}$ è dunque sufficiente provare che:
\begin{equation}
  f(\fixp{g^2}) \leq{} \fixp{g^2}
  \label{hw16-pt2-1}
\end{equation}

Innanzitutto è possibile notare che tale tesi è sensata da dimostrare:
$g \circ g$ è continua (lemma 4.35) sul CPO $D$ e di conseguenza ha minimo
punto fisso $\fixp{g^2}\in D$.

Sviluppando il lato destro di \ref{hw16-pt2-1}, si ottiene:
\begin{equation}
  f(\fixp{g^2}) \leq{} g(g(\fixp{g^2}))
  \label{hw16-pt2-2}
\end{equation}

È noto che $\fixp{g^2}\in D$. È quidi possibile sviluppare la seguente catena
di relazioni:
$$
\begin{array}{lr}
x \leq f(x) & \text{(per ipotesi in $\forall{x}\in D.x \leq{} f(x)$)} \\
f(x) \leq g(x) & \text{(per $\forall{x}\in D. f(x) \leq{} g(x)$)} \\
g(x) \leq f(g(x)) & \text{(perchè $g(x)\in D$)} \\
f(g(x)) \leq g(g(x)) & \text{(per $\forall{x}\in D. f(x) \leq{} g(x)$)} \\
f(x) \leq g(g(x)) &  \text{(per proprietà transitiva di $\leq$)}
\end{array}
$$

Quindi, fissando $x = \fixp{g^2}$\footnote{ciò è possibile proprio perchè
$\fixp{g^2}\in D$}, è provato che
$$
f(\fixp{g^2}) \leq g(g(\fixp{g^2}))
$$

\end{proof}
