\exercise{Esercizio 10}
{
  Fornire un controesempio o provare la seguente equivalenza semantica:
  $$
  \wbS{b}{S} \eqDS (\wbS{b}{S});(\wbS{b}{S})
  $$
}
{}
In semantica denotazionale gli statement per i quali si vuole provare
l'equivalenza corrispondono alle seguenti funzioni:
\begin{itemize}
  \item \dsCtxt{\wbS{b}{S}}
        $\equiv$
        \fixp{F}
  \item \dsCtxt{(\wbS{b}{S});(\wbS{b}{S})}
        $\equiv$
        \dsCtxt{(\wbS{b}{S})} $\circ$ \dsCtxt{(\wbS{b}{S})}
        $\equiv$
        \fixp{F} $\circ$ \fixp{F}
\end{itemize}
, dove $F \equiv{} \lambda{g}.\cond{b}{g \circ \dsCtxt{S}}{\idDS}$

Quindi, al fine di provare tale equivalenza in semantica denotazionale, è
necessario porre:
\begin{equation}
\fixp{F} = \fixp{F} \circ \fixp{F}
\label{hw10-thesis}
\end{equation}

\begin{proof}

Si applichi il funzionale \fixp{F} ad uno stato $s\in\states$ e si prosegua in
con due dimostrazioni separate a seconda che $(\fixp{F}) s$ si riduca ad uno
stato finale o meno.
$$
\boxed{(\fixp{F}) $s = \undefDS$}
$$

L'uguaglianza è facilmente verificabile grazie alla definizione della funzione
di composizione: infatti la composizione $\undefDS \circ \undefDS$ è anch'essa
\undefDS.

In questo modo si ha $\undefDS = \undefDS$, che è banalmente vero.

$$
\boxed{(\fixp{F}) $s = s'$}
$$

\end{proof}
