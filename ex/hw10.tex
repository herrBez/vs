\exercise{Esercizio 10}
{
  Fornire un controesempio o provare la seguente equivalenza semantica:
  $$
  \wbS{b}{S} \eqDS (\wbS{b}{S});(\wbS{b}{S})
  $$
}
{}
In semantica denotazionale gli statement per i quali si vuole provare
l'equivalenza corrispondono alle seguenti funzioni:
\begin{itemize}
  \item \dsCtxt{\wbS{b}{S}}
        $\equiv$
        \fixp{F}
  \item \dsCtxt{(\wbS{b}{S});(\wbS{b}{S})}
        $\equiv$
        \dsCtxt{(\wbS{b}{S})} $\circ$ \dsCtxt{(\wbS{b}{S})}
        $\equiv$
        \fixp{F} $\circ$ \fixp{F}
\end{itemize}
, dove $F \equiv{} \lambda{g}.\cond{b}{g \circ \dsCtxt{S}}{\idDS}$

Quindi, al fine di provare tale equivalenza in semantica denotazionale, è
necessario dimostrare:
\begin{equation}
\forall s \in \states.(\fixp{F}s = (\fixp{F} \circ \fixp{F})s)
\label{hw10-thesis}
\end{equation}

\begin{proof}

Si applichi il funzionale \fixp{F} ad uno stato $s\in\states$ e si prosegua in
con due dimostrazioni separate a seconda che $(\fixp{F}) s$ si riduca ad uno
stato finale o meno.
$$
\boxed{(\fixp{F}) $s = \undefDS$}
$$

L'uguaglianza è facilmente verificabile grazie alla definizione della funzione
di composizione: infatti la composizione 
$$
((\fixp{F}) \circ (\fixp{F})) s = \fixp{F}(\fixp{F}\text{ }s) = \fixp{F}(\undefDS) = \undefDS
$$


In questo modo si ha $\undefDS = \undefDS$, che è banalmente vero.

$$
\boxed{\exists s' \in \states.((\fixp{F}) s = s')}
$$

Grazie all'esercizio 9, si ha che se \fixp{F} $s = s'$ allora
\B{b}$' = \semfalse$. Quindi:
$$
\begin{array}{lr}
(\fixp{F} \circ \fixp{F})\ s = & \text{(Composizione)} \\
(\fixp{F} (\fixp{F}\ s)) = & \text{(Assunzione)} \\
\fixp{F}\  s' = & \text{(Applicazione punto fisso)}\\
F(\fixp{F})\ s' = & \text{(Definizione di $F$)} \\
\cond{b}{\fixp{F} \circ \dsCtxt{S}}{id}\ s' = & (\B{b}' = \semfalse) \\
id\ s' & \text{(Definizione di \idDS)} \\
s'
\end{array}
$$



Dal momento che entrambi i lati dell'equivalenza denotazionale sono uguali ad
$s'$, la tesi è provata.

\end{proof}
