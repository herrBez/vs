\addtocontents{toc}{\protect\setcounter{tocdepth}{1}}

\exercise{Exercise 1}
{Provare o fornire un controesempio per la seguente equivalenza semantica:
\begin{align*}
\wbS{b}{S_1} \eqSOS \wbS{b}{(\ifABC{b}{S_1}{S_2})}
\end{align*}}
{}

L'esercizio va risolto dimostrando la seguente asserzione:

$\forall{s,s'} \in \states.
  (\confSs{\wbS{b}{S_1}}{s} \Rar{*} s')
  \iff
  (\confSs{\wbS{b}{(\ifABC{b}{S_1}{S_2})}}{s} \Rar{*} s')
$

Dunque, fissando due stati $s$ ed $s'$ in \states, è possibile discriminare due
casi in funzione della valutazione di $b$ nello stato $s$.

\subsection{\B{b} = \semfalse}
I due statement vengono derivati nel seguente modo:

\begin{itemize}
  %%%%%%%%%%%%%% first derivation
  \item $\wbS{b}{S_1}$
\begin{align*}
\confSs{\wbS{b}{S_1}}{s} \Rar{}
\whileSOS
\end{align*}
\begin{align*}
\confSs{\ifABC{b}{(S_1;\wbS{b}{S_1})}{\skipForFriends}}{s} \Rar{}
\ifffSOS
\end{align*}
\begin{align*}
\confSs{\skipForFriends}{s} \Rar{}
\skipSOS
\end{align*}
\begin{align*}
s \equiv s'
\end{align*}
  %%%%%%%%%%%%%% second derivation
  \item $\wbS{b}{(\ifABC{b}{S_1}{S_2})}$
\begin{align*}
\confSs{\wbS{b}{(\ifABC{b}{S_1}{S_2})}}{s} \Rar{}
\whileSOS
\end{align*}
\begin{align*}
\confSs{\ifABC{b}
              {(\ifABC{b}{S_1}{S_2}; \wbS{b}{(\ifABC{b}{S_1}{S_2})})}
              {\skipForFriends}}
       {s} \Rar{}
\ifffSOS
\end{align*}
\begin{align*}
\confSs{\skipForFriends}{s} \Rar{}
\skipSOS
\end{align*}
\begin{align*}
s \equiv s'
\end{align*}
\end{itemize}

\subsection{\B{b} = \semtrue}

Nel caso in cui $b$ venga valutato a $\semtrue$ nello stato di memoria $s$, è
banale dimostrare che i due statement sono semanticamente equivalenti. Infatti,
dal momento che la regola $\ifttSOS{}$ non modifica lo stato della memoria,
possiamo tranquillamente asserire che nel caso in cui \B{b} = \semtrue, anche il
costrutto condizionale all'interno del while sceglierà il primo ramo.

Ciò vuol dire che dopo una (o due) valutazioni della guardia $b$ in entrambi i
casi si arriverà rispettivamente ai seguenti statement:
\begin{itemize}
  \item partendo da $\wbS{b}{S_1}$
\begin{align*}
\confSs{(S_1;\wbS{b}{S_1})}{s}
\end{align*}
  \item partendo da $\wbS{b}{(\ifABC{b}{S_1}{S_2})}$
\begin{align*}
\confSs{(S_1; \wbS{b}{(\ifABC{b}{S_1}{S_2})})}{s}
\end{align*}
\end{itemize}

A questo punto viene dimostrato che entrambi i termini convergono allo stesso
stato:

\paragraph{$(\Rightarrow)$}

Si assume che $\confSs{\wbS{b}{S_1}}{s} \Rar{*} s'$.
Perciò, per il lemma di decomposizione, si ha che:

\begin{itemize}
  \item \confSs{S_1}{s} $\Rar{*} s_2$
  \item \confSs{\wbS{b}{S_1}}{s_2} $\Rar{*} s'$
\end{itemize}

Ma allora, per il lemma di composizione, si ha:
\begin{align*}
\confSs{S_1; \wbS{b}{(\ifABC{b}{S_1}{S_2})}}{s}
\Rar{*}
\confSs{\wbS{b}{(\ifABC{b}{S_1}{S_2})}}{s_2}
\end{align*}

Ora abbiamo ottenuto nuovamente i due statement iniziali che partono da uno
stato $s_2$ possibilmente diversi. Tuttavia, le derivazioni che partono da
queste configurazioni sono strettamente più brevi rispetto a quelle iniziali.

Dunque, questo caso è dimostrato per induzione sulla lunghezza della
derivazione.

\paragraph{$(\Leftarrow)$}

La dimostrazione è del tutto analoga al caso $(\Rightarrow)$ appena esposto.

\addtocontents{toc}{\protect\setcounter{tocdepth}{2}}
