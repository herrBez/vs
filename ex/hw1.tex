\addtocontents{toc}{\protect\setcounter{tocdepth}{1}}

\exercise{Exercise 1}
{Provare o fornire un controesempio per la seguente equivalenza semantica:
\begin{align*}
\wbS{b}{S_1} \eqSOS \wbS{b}{(\ifABC{b}{S_1}{S_2})}
\end{align*}}
{}

L'esercizio va risolto dimostrando la seguente asserzione:

$\forall{s,s'} \in \states.
  (\confSs{\wbS{b}{S_1}}{s} \Rar{*} s')
  \iff
  (\confSs{\wbS{b}{(\ifABC{b}{S_1}{S_2})}}{s} \Rar{*} s')
$

Dunque, fissando due stati $s$ ed $s'$ in \states, è possibile discriminare due
casi in funzione della valutazione di $b$ nello stato $s$.

\subsection{\B{b} = \semfalse}
I due statement vengono derivati nel seguente modo:

\begin{itemize}
  %%%%%%%%%%%%%% first derivation
  \item $\confSs{\wbS{b}{S_1}}{s} \Rar{*} s'$
\begin{align*}
\confSs{\wbS{b}{S_1}}{s} \Rar{}
\whileSOS
\end{align*}
\begin{align*}
\confSs{\ifABC{b}{(S_1;\wbS{b}{S_1})}{\skipForFriends}}{s} \Rar{}
\ifffSOS
\end{align*}
\begin{align*}
\confSs{\skipForFriends}{s} \Rar{}
\skipSOS
\end{align*}
\begin{align*}
s \equiv s'
\end{align*}
  %%%%%%%%%%%%%% second derivation
  \item $\confSs{\wbS{b}{(\ifABC{b}{S_1}{S_2})}}{s} \Rar{*} s'$
\begin{align*}
\confSs{\wbS{b}{(\ifABC{b}{S_1}{S_2})}}{s} \Rar{}
\whileSOS
\end{align*}
\begin{align*}
\confSs{\ifABC{b}
              {(\ifABC{b}{S_1}{S_2}; \wbS{b}{(\ifABC{b}{S_1}{S_2})})}
              {\skipForFriends}}
       {s} \Rar{}
\ifffSOS
\end{align*}
\begin{align*}
\confSs{\skipForFriends}{s} \Rar{}
\skipSOS
\end{align*}
\begin{align*}
s \equiv s'
\end{align*}
\end{itemize}

\subsection{\B{b} = \semtrue}

Nel caso in cui $b$ venga valutato a $\semtrue$ nello stato di memoria $s$, è
banale dimostrare che i due statement sono semanticamente equivalenti. Dal 
momento che la regola $\ifttSOS{}$ non modifica lo stato della memoria
possiamo asserire che, nel caso \B{b} = \semtrue, lo 
statement condizionale interno al while eseguirà sempre il primo ramo.

\begin{itemize}
  \item partendo da $\wbS{b}{S_1}$
  $$
\begin{array}{lr}
\confSs{\wbS{b}{S_1}}{s} \Rar{} &
\whileSOS \\
\confSs{\ifABC{b}{(S_1;\wbS{b}{S_1})}{\skipForFriends}}{s} \Rar{} &
\ifttSOS \\
\confSs{S_1;\wbS{b}{S_1}}{s} &
\end{array}
  $$
  \item partendo da $\wbS{b}{(\ifABC{b}{S_1}{S_2})}$
  $$
\begin{array}{lr}
\confSs{\wbS{b}{(\ifABC{b}{S_1}{S_2})}}{s} \Rar{} &
\whileSOS \\
\confSs{\ifABC{b}
              {(\ifABC{b}{S_1}{S_2}; \wbS{b}{(\ifABC{b}{S_1}{S_2})})}
              {\skipForFriends}}
       {s} \Rar{} &
\ifttSOS \\
\confSs{\ifABC{b}{S_1}{S_2}; \wbS{b}{(\ifABC{b}{S_1}{S_2})}}
       {s} \Rar{} &
\ifttSOS \\
\confSs{S_1; \wbS{b}{(\ifABC{b}{S_1}{S_2})}}{s}
\end{array}
  $$
\end{itemize}

Dimostriamo che entrambi i termini convergono allo stesso
stato:

\paragraph{$(\Rightarrow)$}

Se $\confSs{\wbS{b}{S_1}}{s} \Rar{*} s'$, allora
$\confSs{S_1;\wbS{b}{S_1}}{s} \Rar{*} s'$.

Per il lemma di decomposizione si ha che $\exists s'' \in \states$ tale che:
\begin{itemize}
  \item \confSs{S_1}{s} $\Rar{*} s''$
  \item \confSs{\wbS{b}{S_1}}{s''} $\Rar{*} s'$
\end{itemize}

Ma allora, per il lemma di composizione, si ha:
\begin{align*}
\confSs{S_1; \wbS{b}{(\ifABC{b}{S_1}{S_2})}}{s}
\Rar{*}
\confSs{\wbS{b}{(\ifABC{b}{S_1}{S_2})}}{s''}
\end{align*}

Ora abbiamo ottenuto nuovamente i due statement iniziali che, anzichè
cominciare l'esecuzione su $s$, partono da uno stato $s''$:
$$
\confSs{\wbS{b}{S_1}}{s''} \Rar{*} s'
  \implies
\confSs{\wbS{b}{(\ifABC{b}{S_1}{S_2})}}{s''} \Rar{*} s'
$$

Dal momento che le derivazioni che partono da queste configurazioni sono
strettamente più brevi rispetto a quelle iniziali\footnote{questo poichè lo
stato $s''$ è stato raggiunto tramite alcuni passi di derivazione}, questo
caso è dimostrato per induzione sulla lunghezza della derivazione.

\paragraph{$(\Leftarrow)$}

Rimane da dimostrare che se
$\confSs{S_1; \wbS{b}{(\ifABC{b}{S_1}{S_2})}}{s} \Rar{*} s'$, allora \\
$\confSs{S_1; \wbS{b}{S_1}}{s} \Rar{*} s'$.

Per il lemma di decomposizione si ha che $\exists s'' \in \states$ tale che:
\begin{itemize}
  \item \confSs{S_1}{s} $\Rar{*} s''$
  \item \confSs{\wbS{b}{(\ifABC{b}{S_1}{S_2})}}{s''} $\Rar{*} s'$
\end{itemize}

Ma allora, per il lemma di composizione, si ha:
\begin{align*}
\confSs{S_1; \wbS{b}{S_1}}{s}
\Rar{*}
\confSs{\wbS{b}{S_1}}{s''}
\end{align*}

Ora abbiamo ottenuto nuovamente i due statement iniziali che, anzichè
cominciare l'esecuzione su $s$, partono da uno stato $s''$:
$$
\confSs{\wbS{b}{S_1}}{s''} \Rar{*} s'
  \impliedby
\confSs{\wbS{b}{(\ifABC{b}{S_1}{S_2})}}{s''} \Rar{*} s'
$$

Dal momento che le derivazioni che partono da queste configurazioni sono
strettamente più brevi rispetto a quelle iniziali\footnote{questo poichè lo
stato $s''$ è stato raggiunto tramite alcuni passi di derivazione}, questo
caso è dimostrato per induzione sulla lunghezza della derivazione.

\cvd

\addtocontents{toc}{\protect\setcounter{tocdepth}{2}}

