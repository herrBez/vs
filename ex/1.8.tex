\exercise{1.8}
{Provare che le equazioni della tabella 1.1 definisce una funzione totale $\mathcal{A}:\aexp \rightarrow \states \rightarrow \num$: 
	\begin{enumerate}
		\item Argomenta che è sufficiente provare che per ogni $a \in \aexp$ e ogni $s \in \states$
		c'é esattamente un valore $\mathbf{v} \in \mathbf{Z}$ tale che $\A{a} = \mathbf{v}$
		\item Usa l'induzione strutturale sulle espressioni aritmetiche per provare che è effettivamente
		così.
	\end{enumerate}
}
{
	\begin{enumerate}
		\item Siccome per i casi base, ovvero
		\begin{itemize}
			\item $\boxed{a = n}$ per un numerale $n$ qualsiasi
			\item $\boxed{a = x}$ per una variabile $x$ qualsiasi
		\end{itemize} 
		abbiamo che rispettivamente:
		\begin{itemize}
			\item $\boxed{\A{n} \mapsto \N{n}}$ che sappiamo essere un valore $\mathbf{v}$ univoco, dato che $\mathcal{N}$ è una funzione che mappa numerali in numeri.
			\item $\boxed{\A{x} \mapsto s\myspace x}$, visto che $s$ è una funzione (totale) $\var \rightarrow \num$  allora
			alla variabile $x$ verrà associato un unico valore $\mathbf{v} \in \num$.
			
			Le espressioni composite sono formate da un operatore binario e due sottoespressioni
			$a_1$ e $a_2$.
			Sia $a_1$ che $a_2$ vengono associate ad un unico valore $\mathbf{v_1}, \mathbf{v_2} \in \num$.
			Siccome gli operatori semantici $+, -, *$ per definizione prendono in input due
			valori e ne restituiscono rispettivamente la somma, la differenza e il prodotto, che
			non sono altro che valori, l'asserto è dimostrato.
		
			
		\end{itemize}
			\item Procedo per induzione strutturale su \aexp, formalizzando quanto argomentato
			in precedenza.
			\begin{table}[h!]
				\begin{center}
			\begin{tabular}{| l| L | p{0.4\linewidth} |}
				\hline
				\textbf{Caso Base} & \A{n}=\N{n}=\mathbf{v} & 
				è banalmente vero perchè \Ncal associa ad un numerale un (unico) numero per definizione.\\
				& &\\
				& \A{x} = s \myspace x = \mathbf{v}&$s$ è una funzione che ho come dominio tutte le variabile e quindi ritorna un valore univoco $\mathbf{v}$ dipendentemente dallo stato.\\
				& & \\
				\hline
				& &\\
				\textbf{Casi Compositi} & \A{a_1 + a_2} = \A{a_1} + \A{a_2} & Definizione di \A{a_1 + a_2}\\
				& = \mathbf{v_1} + \mathbf{v_2}  &  Ipotesi induttiva applicata a \A{a_1} e \A{a_2}\\
				& = \mathbf{v} & Definizione dell'operatore semantico\\
				& & \\
				& \A{a_1 + a_2} = \A{a_1} - \A{a_2} & Definizione di \A{a_1 - a_2}\\
				& = \mathbf{v_1} - \mathbf{v_2}  &  Ipotesi induttiva applicata a \A{a_1} e \A{a_2}\\
				& = \mathbf{v} & Definizione dell'operatore semantico\\
				& & \\
				& \A{a_1 + a_2} = \A{a_1} * \A{a_2} & Definizione di \A{a_1 * a_2}\\
				& = \mathbf{v_1} * \mathbf{v_2}  & Ipotesi induttiva applicata a \A{a_1} e \A{a_2}\\
				& = \mathbf{v} & Definizione dell'operatore semantico\\
				& & \\
				\hline
			\end{tabular}
				
				\end{center}
			\end{table}
			
	\end{enumerate}
}
