\exercise{1.8}
{Provare che le equazioni della tabella 1.1 definisce una funzione totale $\mathcal{A}:\aexp \rightarrow \states \rightarrow \num$: 
	\begin{enumerate}
		\item Argomenta che è sufficiente provare che per ogni $a \in \aexp$ e ogni $s \in \states$
		c'é esattamente un valore $\mathbf{v} \in \mathbf{Z}$ tale che $\A{a} = \mathbf{v}$
		\item Usa l'induzione strutturale sulle espressioni aritmetiche per provare che è effettivamente
		così.
	\end{enumerate}
}
{
	\begin{enumerate}
		\item Siccome per i casi base, ovvero
		\begin{itemize}
			\item $\boxed{a = n}$ per un numerale $n$ qualsiasi
			\item $\boxed{a = x}$ per una variabile $x$ qualsiasi
		\end{itemize} 
		abbiamo che rispettivamente:
		\begin{itemize}
			\item $n \mapsto \N{n}$ che sappiamo essere un valore $\mathbf{v}$ univoco, dato che $\mathcal{N}$ è una funzione che mappa numerali in numeri.
			\item $s  x$
			\N
			
		\end{itemize}
	\end{enumerate}
}
